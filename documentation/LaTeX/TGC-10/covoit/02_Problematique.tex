Si le co-voiturage existe déjà entre ces 3 centres, il reste très limité. Par exemple seul le site de l'INRIA, propose des liens vers différents sites de covoiturage:\\

 \textit{http://www-sop.inria.fr/presentation/localisation\_fr.shtml} \\

Or tous ces sites ne sont pas forcement adaptés a la demande très localisé de ces trois centres. De plus il existe des situations dans lesquelles
les deux catégories de personnes présentes dans cette zone ne souhaite pas forcement faire du covoiturage ensemble (par exemple professeurs et étudiants en période d'examens... ) 
Au final il est très rare de voir un étudiant faire du covoiturage avec un autre membre non étudiant d'un des 3 centres de recherches. \\

Plusieurs problèmes doivent donc être résolut: \\
Dans un premier temps, proposer un service de co-voiturage localisé, personnalisé et adapté a chaque catégorie, afin d'inciter son utilisation. C'est a dire un service visant les étudiants d'une part, et un service visant les enseignants/chercheurs et autres personnels d'autres part.\\
Dans un deuxième temps le principal problème a résoudre ici est d'inter-connecter ces deux mondes/réseaux hétérogènes, pourtant très proche géographiquement, et ainsi améliorer encore plus l'impacte du co-voiturage dans cette zone. 




