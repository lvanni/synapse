Among the potential improvement, we shortly mention a few ones.
%
\subsection{Improved network bootstrap and community discovery}
%
At the present state a new community can be setup or joint by passing
through the tracker. This keeps track of community activities, their
location, handles the join negotiation and restrictions and can
suggests nearby communities that could be joined, however it
constitutes also a centralized point of failure for all the
communities. To improve further more the mechanism the following
solutions can be put in place:
%
\begin{itemize}
\item Assuming that a community/overlay could very likely reside on
  the same network infrastructure (i.e. the enterprise intranet) a
  discovery protocol can be put in place leveraging existing
  technologies like Avahi \cite{avahi:website} to discover new peers
  or new networks to join;

\item Peer caching could be used to reconnect to previously connected
  peers whose activity is known to be reliable;

\item An invitation to a new community could be handled physically via
  an NFC transaction. A user with an NFC enabled phone could be
  invited by another user by simply swiping the phones
  together. Furthermore this could be an additional guarantee of a
  user ``reliability'' as the new participant needs to be known and
  met by an existing member;

\item The community database could be as well stored in the DHT
  itself, meaning by this that new communities could simply be
  discovered through specific requests routed through existing
  synapses to other networks in a ping-like way.
\end{itemize}


\subsection{Semantic queries and specialized protocols}\label{sec:semantic}
%
It appears clear that the current approach suffers from the limitation
of a simple key-value approach. Such approach does not fit well into
an application that finds her strength in the possibility of
performing searches according to many different criteria.  The
adoption of a semantic hash function (such as \cite{SemanticHash})
would allow to cluster in nearby peers information semantically close
(\ie\ trips heading to sibling destinations or taking place in the
same time window).  Needless to say, with such hashing in place the
adoption of an overlay protocol more suited to range queries (like
P-Ring \cite{P-Ring}, P-Grid\cite{P-Grid} or Skipnet
\cite{Harvey03skipnet:a}) might lead the semantically significant
range queries, where, for example, departure and arrivals can be
geographically mapped and queried with a certain range in Km.

Another possible improvement (currently under study) would be to use a
DHT protocol more suited for geo-located information. CAN \cite{CAN})
in a 2D configuration is a first example of how this could be
achieved. Mapping CAN's Cartesian space over a limited geographic area
(like in Placelab \cite{PlaceLab}) could ease the query routing and
eventually provide some strategic points to place synapsing nodes.

\subsection{Overlay-underlay mapping optimizations}
%
From a networking point of view, the scientific hard point to solve is
the overlay-underlay network mapping (to avoid critical latency issues
due to the fact that one ``logical'' hop may via ``many'' physical
hops); The issue is being investigated and could involve for example
the use of several always on-servers to be used to ``triangulate'' the
position of a peer over the internet (according to latency metrics)
and cluster together ``nearby'' peers (whereas nearby will mean sharing
similar latencies to the same given servers).  Another issue would be
to make the service firewall-resilent by implementing TCP Punch-hole
techniques in the peer engine and in the tracker.


\subsection{Backward compatibility with other Carpool services}
%
At the present time we have developed a CarPal service that is
suitable to interconnecting Carpool services whose standards are open,
collaborative and based on our software. Unfortunately this is not
always the case. This makes a backward compatibility problem in case
we want to connect existing services. To take into account this case,
the Synapse protocol we have developed allows also a, so called,
\emph{black box} variant, that is suitable to interconnecting overlays
that, for different reasons, are not collaborative at all, in the
sense that they only route packets according to their proprietary and
immutable protocol.  Being the black box more of a meta-protocol
running on top of existing, and not necessarily peer-to-peer,
protocols, we can imagine strategically placed Synapse nodes being
responsible of querying existing web services and returning the
corresponding information as if they were coming from a foreign
network. This open new scenarios were multimodality is easily
integrated and made available for the nearby communities. The system
needs to be properly designed in order to avoid a situation where too
many peers act as a Distributed Denial Of Service, but the current
infrastructure make it easily feasible.

\subsection{User rating, social feedback}
%
Being CarPal an application based on user-generated content and
designed to put in touch people not necessarily acquainted to each
other, it is important to implement some social feedback and security
mechanism to promote proactive and good behaviors by the users.
Borrowing from today's most successful web applications, 2 solutions
can be imagined:
%
\begin{itemize}
\item A user rating should be put in place in order, for example, to
  allow passengers to evaluate a driver's punctuality, behavior and
  driving skills, and vice-versa. This feedback, similar to what
  systems like Ebay already have since several years, can help
  maintaining a high level of quality in the service by giving an
  immediate picture of a driver or passenger's reliability;

\item Some points can be assigned to users based on their activity in
  the community. The more a user will be proactive by publishing or
  subscribing to new trips in an overlay, the more ``karma points'' he
  will receive.  A similar approach can be verified in Social News
  website like Digg~\cite{Digg} or Reddit~\cite{Reddit} and has become
  pretty common in today's social media.  With the deployment and
  integration of new distributed services, these points could act as a
  ``virtual cash'' and grant access to features normally reserved to
  paying customers, thus motivating drivers and passengers to keep a
  community alive.
\end{itemize}

\subsection{Other potential applications}
%
The Car sharing/pooling is not the exclusive applicative field the
overlay network technology we designed; with the same final objective
of minimizing traffic, pollution and energy a service interconnecting
transportation companies Information Systems could be envisaged. A
``BoxPal'' system could be easily build using the same overlay network
technology: the only difference being the (more difficult) 3D
bin-packing combinatorial algorithms employed instead of simple
matching of drivers/cars/itinerary/car places.

%%%%%
%%%%% SPARE MATERIAL  C O M M E N T !!!
%%%%%
\comment{ TUTTO RFID MATERIAL IN VRAC DA POMPARE....VEDI ANCHE NEL PDF
CHE TI HO MANDATO
%
Another potential application would be the introduction of the RFID
technologies that could be used to collect information about locations
and possessions of people (and hence about their preferences, possible
medical conditions, ...) without them knowing about it generated fears
in the general public.
%
\Gi{DS = discovery services}
%
The usability in the Enterprise Supply Chain is an immediate proof of
usage of our approach. From a networking point of view, the scientific
hard point to solve are the mapping overlay-underlay network (to avoid
critical latency issues due to the fact that one ``logical'' hop may
via ``many'' physical hops); - crossing firewalls; - merging different
overlay networks (\eg\ , networks of different companies); -
distributing semantic queries toward a P2P network; - trading off
between routing complexity and of routing method for exhaustive
discovery;- dealing with heterogeneous underlay networks (IP, Wifi,
Zigbee, GPRS), just to mention a few.
%
Our overlay technology has various originalities that render it very
interesting from a social viewpoint. First, its particular knowledge
sharing approach will permit databases and DSs to support much more
cooperation between people than currently. Second, it will also permit
people to design their own access control or usage rules and associate
them to the information they own. Third, thanks to our DSs, people (or
their personal agents) will better be able to find which databases
have stored information about them (then, if these persons are not
happy with that storage, they can legally ask a database
server/manager to remove or update such information). Fourth,
conversely, if these persons have their own personal databases where
they have described access control and usage rule for some kinds of
information about themselves (\eg, their age or their weight), thanks
to our DS, database servers could find such rules and respect
them. This last point is certainly a naive ideal but could be checked
by people thanks to the third point and then made well known by them
or rewarded in some other way.
%
Systems such as ``Google Latitude'' allows people to let persons they
trust track them. A flexible and secure DS could be uhypothesis.
%
 One objective of RFID3S is to build a P2P overlay network where peers
contain fragments of RFID databases. Due to the heterogeneity of the
network and due to the presence of peers playing different roles and
of SW and HW barriers between companies (very often competitors) we
plan to use a - so called - hybrid overlay networks. A popular
topology has a super-peer serving as a server for a subset of the
peers, as in well know Skype VOIP application.
%
Towards a discovery service in a scalable, secure and flexible
Internet of things including RFID and NFC devices.  The final
objective of RFID3S is to build a hybrid overlay network where
security, scalability and semantic expressivity of queries cohabit
without tears for companies and end-users. Using a P2P infrastructure
for modeling an ONS of RFID/NFC in presence of different kinds of
peers and different kind of underlay network is probably one of the
more interesting challenges of the project.  Routing also complex
queries towards this network in a secure way is also worth of
interest.
%
The usability in the Enterprise Supply Chain is an immediate proof of
usage of our approach. From a networking point of view, the scientific
hard point to solve are the mapping overlay-underlay network (to avoid
critical latency issues due to the fact that one ``logical'' hop may via
``many'' physical hops); - crossing firewalls; - merging different
overlay networks (e.g., networks of different companies); -
distributing semantic queries toward a P2P network; - trading off
between routing complexity and of routing method for exhaustive
discovery;- dealing with heterogeneous underlay networks (IP, Wifi,
Zigbee, GPRS), just to mention a few.  }



\comment {spostato tutto in fondo. Cominciamo dal tecnico e andiamo sul funzionale.}

\comment{Questo te l' ho commentato perche' o i concetti sono gia'
  espressi sotto (reti eterogenee, firewall, mapping etc) o sono
  concetti che andrebbero sviluppati e fatti aderire, e io ora non
  riesco davvero a vedere come la supply chain possa essere pertinente
  a un servizio di covoiturage.}

\comment{Leading from the above would be the introduction of RFID
  technologies that could be used to collect information about
  transported goods.  The usability in the Enterprise Supply Chain is
  an immediate proof of usage of our approach. From a networking point
  of view, the scientific hard point to solve are the mapping
  overlay-underlay network (to avoid critical latency issues due to
  the fact that one "logical" hop may via "many" physical hops); -
  crossing firewalls; - merging different overlay networks (\eg\,
  networks of different companies); - distributing semantic queries
  toward a P2P network; - trading off between routing complexity and
  of routing method for exhaustive discovery;- dealing with
  heterogeneous underlay networks (IP, Wifi, Zigbee, GPRS), just to
  mention a few.  Another potential application would be in the
  provisioning of supermarket companies.}
