\subsection{Application principles}
%
One of the most important features for a car share application is to
be able to maximize the chances of finding a match between 2 users (1
driver and 1 or more travelers).  From this comes the choice of
arranging the database by communities in order to put in touch people
who most likely share the same traveling patterns in space and time
(\eg\ work for the same company, attend the same university and so
on). Another important aspect is to be able to update the planned
itinerary information as quick as possible so that a last minute
change in plans can be easily managed and updated and may eventually
lead in finding a new match.

For the above reasons CarPal has been intended as a desktop and mobile
application running on a peer-to-peer overlay network. This allows a
community of people to spontaneously create its own travel DB (which,
as it will be shown later, can be interconnected with siblings
communities) and manage it in a distributed manner.  Moreover it
constitutes a flexible infrastructure where,by being deployed on
connected mobile devices, it will be possible to develop more advanced
info-mobility solutions that might take into account the position of
the user/vehicle (via the internal GPS), geographically-aware network
discovery or easy network join or vehicle tracking through checkpoints
with the use of NFC technologies~\cite{NFC}.

\subsection{Data structure}
%
The combinations are then stored in the DHT according to the following table:
\begin{enumerate}
\item All the information concerning the trip are stored as
  \key{INFO} using a key obtained by hashing a newly generated
  \key{Trip\_ID}.  The information stored concerns trip date and time,
  departure and arrival places, if the trip has to be repeated in
  time, number of available seats (or cargo space, in case of shared
  goods transportation), user contact and other relevant information.
\item a chained list of INFO is created (or updated if already
  existing) as \key{BY\_TIME\_DEP\_ARR}, containing all the
  trips for a certain itinerary at a certain moment.
\item \key{BY\_DAY\_DEP\_ARR} is also updated. It contains all the trips for a certain itinerary at a certain date (useful to
quickly query all the trips of the day on a certain itinerary).
\item \key{BY\_DAY\_DEPARTURE} and \key{BY\_DAY\_ARRIVAL} are 2 lists that contains the references of the trips arranged by day and by departure or destination. This allows, for example, to quickly see who at a certain day will be going, say, to the airport or will be leaving from Company X.
\item \key{BY\_DAY} contains all the BY\_DAY\_DEP\_ARR relative to a given date. Useful to browse through all the 
available trips in a certain day.
\item \key{BY\_DEPARTURE} and \key{BY\_ARRIVAL} also contain a list of BY\_DAY\_DEP\_ARR arranged by
departure place or arrival place. 
\item \key{BY\_USER} lists the INFO by UserID, used mostly for maintenance purposes and data updates.
\end{enumerate}

\begin{table}
	\begin{center}
\key{
		\begin{tabular}{|l|l|l|}
			\hline
			\key{Type} & \key{Key} & \key{Value} \\
			\hline
			TRIP & ID & USER,PLACE,TIME,SEATS,INFO \\
			\hline
			BY\_TIME\_DEP\_ARR & DATE,DEP,TOD,ARR,TOA & list[TRIP] \\
			\hline
			BY\_DAY\_DEP\_ARR & DATE,DEP,ARR& list[TRIP] \\
			\hline
			BY\_DAY\_DEP & DATE,DEP & list[TRIP] \\
			\hline
			BY\_DAY\_ARRIVAL & DATE,ARRIVAL & list[TRIP] \\
			\hline
			BY\_USER & USERID & list[TRIP] \\
			\hline
		\end{tabular}
}
	\end{center}
	\caption{Different data structures stored in the DHT}
\end{table}

	
As we can see, the intention here is to give the maximum search flexibility to the user to find his trip.
We believe in fact that one of the key problems of car sharing is the high heterogeneity of offer and demand.
Such heterogeneity makes extremely difficult for an automatic system to find and manage a satisfactory number of matches.

Therefore, we opted to give different search possibilities to the user, from fine grained to a broader search, and leave to
his intelligence the remaining information skimming.

\Vi{au cas ou rajouter des examples, ici ou ailleurs.}

\subsubsection{Data approximation}
Geographic and time information must be encoded in such a way to be precise enough to be 
still relevant for our purposes (someone leaving from the same city but 10 km far is not a useful match)
yet broad in a way that a punctual query will not skip important results.

Until a feasible way to perform semantically relevant range queries will be found, departure/arrival hotspots
can be added in a social manner. A driver departing from a certain spot for the first time will be offered a map 
where to mark his spot of departure, inviting the user to use an already defined spot nearby in case.

Concerning time approximation, a 20 minutes window is used to approximate departure times. Both
during an insertion or a query, anything within the 0-19 minutes interval will be automatically 
set at 10 minutes, 20-39 will be set at 30 minutes and 40-59 at 50.

\subsection{Real time updates}
When a user wants to find a car, he can perform a search using one or more of the criteria provided.
Once a suitable trip has been found, a negotiation can be started by directly contacting the driver who 
posted the trip. Such negotiation, being either simple messages or some more elaborate
procedure, is foreseen in order to allow the 2 parties to exchange relevant information about the trip
(e.g. urgency, expenses reimbursement, pickup place)

If the 2 parties come to an agreement, the seats/cargo (in case of goods) field is updated in the DHT
and the user subscribes to the drivers peer in order to get real time updates about its conditions.
Such updates might involve broadcast messages (the driver will be sick today, slightly late, had a complication)
or automatic position updates. Message dispatch should be handled by a publish-subscribe system.

\subsection{Network architecture}
%
The overlay chosen for the proof of concept is Chord\cite{chord}
although other protocol could be used to leverage the locality of the
application (\ie\ SkipNet\cite{skipnet} or a more direct geographical
mapping using a CAN \cite{CAN}.  Even on a simple Chord mechanisms to
ensure fault tolerance can be put in place, like data replication
using multiple hash keys or request caching.

To allow a new community to be start up, a \emph{public tracker} has
been put in place on the Internet, keeping track of all the
communities and relevant information about them, like localization,
brief description, recent activities (to identify inactive
communities), if the access is public or upon invitation and one or
more entry peers, in a YODL-like fashion~\ref{YODL}.