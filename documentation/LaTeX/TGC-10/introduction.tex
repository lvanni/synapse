%
%
%
\subsection{Context}
%
Car pooling is the shared use of a driver's personal car with one or
more passengers, usually but not exclusively colleagues or friends,
for commuting (usually small-medium recurring trips, like \eg\ home to
work, home to school, you name it). Amongst the many advantages it
decreases traffic congestion and pollution, reduces trip expenses
by alternating the use of the personal vehicle amongst different
drivers and enables the use of dedicated lanes or reserved parking
places where made available by countries aiming to
reduce the global dependency of petrol.

Car sharing is a model of car rental for short period of time (in
opposite of the classical car rental companies), where a number of
cars, often small and energy-efficient, are spread on a small
territory, like a city. Customers first subscribe to a company who
exploits and maintains the car park, then use those cars for their
personal purposes. Service fees are normally per kilometer and
insurance and fuel costs are included in the rates. Car sharing is an
interesting option for families that need a second car but do not
want to buy it. Modern geolocation technologies using GPS and mobile
phones help to find the closest car to pick. The same
economical/ecological advantage of car pooling applies, and
mathematically speaking they are parameters of the same function we
want to minimize.

\subsection{Problem overview}
%
In Car* services, an Information System (IS) has been showed to be
essential to match the offers, the requests, and the resources. The
Information System is, in most cases, a front-end web site connected
with a back-end database.  A classical client-server architecture is
usually sufficient to manage those services. Users register their
profile to one Information System and then post they
offers/requests. In presence of multiple Car* services, for technical
and/or commercial reasons, it is not possible to share contents across
the different providers, despite the evident advantage. As a simple
example the reader can have a quick look on those two web sites
Equipage06\footnote{\url{http://www.covoiturage-cg06.fr/}.} and Otto
and co\footnote{\url{http://www.ottoetco.org/}.} concerning car
pooling in the French Riviera region. At the moment those two web
sites does not communicate at all and do not share any user profile
neither they share any request, even if they operate on the same
territory and with the same objectives. Since both services are no
profit, the reason for this has probably to be found in the
client-server nature of both Information Systems that, by definition,
are not incline to collaborate with each other. Although in principle
this does not affect the correct behavior of both services, it is
clear that \emph{``In union is strength''}\footnote{Italian
  proverb.}. Moreover, the classical shortcomings of client-server
architectures make both service unavailable in case both servers are
down.


\subsection{Contributions}
%
As main contributions in this paper:
%
\begin{itemize}
\item we design and implement a Peer-to-peer based Carpool information
  system, that we call \emph{CarPal}: this service is suitable to be
  deployed with a very low infrastructure and can run on various
  devices spanning from PC to a small intelligent devices, like
  smartphones;

\item we customize the Arigatoni protocol and his evolution, the
  Synapse protocol, both specialized for resource discovery in overlay
  networks in order to allow two completely independent CarPal-based
  Information systems to communicate without the need of merging one
  CarPal system into the other or, even worse, build a third CarPal
  system including both.
\end{itemize}



\subsection{Outline}
%
The rest of the paper is organized as follows: in
Section~\ref{sec:link} we describe the interconnection of different
CarPal systems by means of the Synapse protocol developed in our
team. In Section~\ref{sec:architecture}, we introduce our CarPal
service and we show how it is mapped onto a Distributed Hash Table.
In Section~\ref{sec:proof} we show a running example with a
proof-of-concept that we have implemented in our team on the basis of
a real case of study in our French Riviera area of Sophia Antipolis a
technological pole of companies and research centers. A GUI is also
presented\footnote{See~\url{http://www-sop.inria.fr/teams/lognet/carpal}.}.
Section~\ref{sec:experiences} describes the deployment of a client
prototype\footnote{See~\url{http://www-sop.inria.fr/teams/lognet/synapse}.}
over the Grid'5000 platform.  Section~\ref{sec:conclusion} concludes
and presents some further work.
