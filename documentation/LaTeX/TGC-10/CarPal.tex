\documentclass{llncs}
\usepackage{times,url}
\usepackage{amsmath,amssymb,amsfonts,bbm}
\usepackage{graphicx,epsfig,epsf}
\usepackage{alltt}
\usepackage{url}
\usepackage{cite}
%\usepackage{babel}
\usepackage[utf8]{inputenc}
\usepackage[T1]{fontenc}
\usepackage{multirow}
 
   
% mix and mac 
%\usepackage[latin1]{inputenc}
\usepackage{algorithm}
\usepackage[usenames]{color}

 

%%%
\newcommand{\ie}{i.e.}
\newcommand{\eg}{e.g.}
\newcommand{\vs}{vs.}
\newcommand{\key}[1]{{\small {\sf #1}}}



\newcommand{\verbi}[1]{{\small\texttt{#1}}}
\newcommand{\comment}[1]{}


%% Intra nossss
\newcommand{\Ce} [1]{{[\color{cyan}{Cdric}}: {#1}]}
\newcommand{\Gi} [1]{{[\color{green}{Gigi}}: {#1}]}
\newcommand{\La} [1]{{[\color{cyan}{Laurent}}: {#1}]}
\newcommand{\Vi} [1]{{[\color{blue}{Vincenzo}}: {#1}]}

\newcommand{\RESP} [1]{{[\color{red}{RESP: #1}]}}




%% High penalties for line and paragraph-breaking [Dan]
\pretolerance=2000 \binoppenalty=2000 \relpenalty=1500
%\interlinepenalty=150 \predisplaypenalty=10000 \postdisplaypenalty=400
\hbadness=5000 \hfuzz=2pt 



%%%% FORMATTING
\newcommand{\rew}[1]   {\hspace{-#1mm}}
\newcommand{\fwd}[1]   {\hspace{#1mm}}
\newcommand{\down}[1]  {\vspace{#1mm}}
\newcommand{\up}[1]    {\vspace{-#1mm}} %je l'ai du decommenter parce que il cassait la compil


\usepackage{algorithm}

\newfloat{algo}{thp}{}
\floatname{algo}{Algorithm}

\newcommand{\R}{\mathtt{root}}

\newcommand{\TRUE}{\mathtt{true}}
\newcommand{\FALSE}{\mathtt{false}}
%\newcommand{\R}{\mathtt{R}}
%\newcommand{\C}{\mathcal{C}}
%\newcommand{\OR}{\mathtt{O}}
\newcommand{\IF}[1]{\textbf{if} {#1}}
\newcommand{\THEN}{\textbf{then} }
\newcommand{\ELSE}{\textbf{else} }
\newcommand{\ELSEIF}[1]{\textbf{elseif} {#1}}
\newcommand{\ENDIF}{\textbf{endif} }
\newcommand{\FOR}[1]{\textbf{for} {#1} \textbf{do }}
%\newcommand{\FOR}[2]{\textbf{for } {#1} \textbf{to } {#2} \textbf{do }}
\newcommand{\FORALL}[1]{\textbf{for all} {#1} \textbf{do }}
\newcommand{\FOREACH}[1]{\textbf{for each} {#1} \textbf{do }}
\newcommand{\WHILE}[1]{\textbf{while} {#1} \textbf{do }}
\newcommand{\REPEAT}{\textbf{repeat} }
\newcommand{\FOREVER}{\textbf{forever} }
\newcommand{\DONE}{ \textbf{done}}

\newcommand{\ALGOHEADER}[3]
{\begin{tabular}[t]{@{\extracolsep{0pt}}p{#3}l}
    \textbf{#1} &
    \begin{tabular}[t]{@{\hspace{15pt}}l}
        #2
    \end{tabular}
 \end{tabular}\smallskip
}

\makeatletter
\newcommand{\alglabel}[1]{%
  \@bsphack%
  \protected@write\@auxout{}%
         {\string\newlabel{#1}{{\the\ALGONum\LineSep \formatLine}{\thepage}}}
  \@esphack%
}
\makeatother

\newcommand{\CST}[1]      {\ALGOHEADER{Constants: }{#1}{.5in}}
\newcommand{\VAR}[1]        {\ALGOHEADER{Variables: }{#1}{.5in}}
\newcommand{\LOCALVAR}[1]   {\ALGOHEADER{Local  Variables: }{#1}{1.2in}}

\newcommand{\MACRO}[1]      {\ALGOHEADER{Macros: }{#1}{.5in}}

%\newcommand{\FUNCTION}[2]   {\textbf{function}  ${#1}$: \textbf{{#2}}}

\newcommand{\FUNCTION}[1]{\textbf{Function} {#1}}

\newcommand{\RETURN}[1]     {\textbf{return} {#1}}
\newcommand{\PROC}[1]       {\textbf{procedure} ${#1}$}
\newcommand{\INIT}[1]       {\textbf{initially} ${#1}$}
\newcommand{\ACTION}        {\textbf{actions:}}

\newcommand{\MAC}[2]
{
   ${#1} \equiv $
   \begin{tabular}[t]{@{\extracolsep{0pt}}l}
       #2
   \end{tabular}
}

\newcommand{\RCV}[1]{\textbf{upon} receipt \textbf{of} $<$#1$>$ \textbf{do}}
\newcommand{\RCVFROM}[2]{\textbf{upon} receipt \textbf{of} $<$#1$>$ \textbf{from} #2 \textbf{do}}
\newcommand{\RCVFROMSYNC}[2]{\textbf{receive} $<$#1$>$ \textbf{from} #2} %\textbf{to} #2}

\newcommand{\SEND}[2]{\textbf{send} $<$#1$>$ \textbf{to} #2}
\newcommand{\SENDSYNC}[2]{\textbf{send-sync} $<$#1$>$ \textbf{to} #2}
\newcommand{\SENDTOHOST}[1]{\textbf{send\_to\_host}$<$#1$>$}
\newcommand{\RCVFROMHOST}[1]{\textbf{receive\_from\_host}$<$#1$>$}

\newcommand{\PROCINIT}{\textbf{upon} INITIALIZATION}

\newcommand{\BEGLIST}{\begin{list}{}{\partopsep -3pt \parsep -2pt \listparindent -0pt \labelwidth .5in}}
\newcommand{\ENDLIST}{\end{list}}
\newcount\ALGOLine
\ALGOLine=-1
\newcount\ALGOLineStart
\ALGOLine=0
\newcount\ALGONum
\ALGONum=1

\newcommand{\LineSep}{.}
\newcommand{\LINESTYLE}{\scriptsize}
\newcommand{\INITALGO}[1]{\global\ALGONum=#1}
\newcommand{\INITLINE}[1]{\global\ALGOLineStart=#1}
\newcommand{\RESETLINE}{\global\ALGOLine=\ALGOLineStart}
\newcommand{\formatLine}{\ifnum\the\ALGOLine<10 0\fi\the\ALGOLine}
\newcommand{\NA}{\global\advance\ALGONum  by 1 \RESETLINE}
\newcommand{\AL}{\global\advance\ALGOLine by 1 \LINESTYLE{$\the\ALGONum$\LineSep$\formatLine$}}
\newcommand{\VL}{\ \vline\>}

\newcommand{\logreq}{{\tt logReq}}
\newcommand{\hostreq}{{\tt hostReq}}
\newcommand{\updatechild}{{\tt updateChild}}
\newcommand{\addchild}{{\tt addChild}}
\newcommand{\scanreq}{{\tt scanReq}}
\newcommand{\replicationreq}{{\tt replicationReq}}
\newcommand{\addparent}{{\tt addParent}}

\newcommand{\commonprefix}{{\bf COMMONPREFIX}}
\newcommand{\sizeof}{{\bf SIZEOF}}
\newcommand{\getpeer}{{\bf GETPEER}}
\newcommand{\getnbreplicas}{{\bf GETNBREPLICAS}}
\newcommand{\getbestreplica}{{\bf GETBESTREPLICA}}

\newcommand{\PREF}[1]{\mbox{{\sc Prefixes}}({#1})}

%%%%% ASYNC REPAIR
\newcommand{\destroy}{{\footnotesize{\bf DESTROY}}}
\newcommand{\prefix}{{\footnotesize{\bf PREFIX}}}
\newcommand{\isprefix}{\sc IsPrefix}
\newcommand{\len}{{\footnotesize{\bf LEN}}}
\newcommand{\gcp}{{\footnotesize{\bf GCP}}}
\newcommand{\inser}{{\footnotesize{\bf INSERT}}}
\newcommand{\newnode}{\sc NewNode}

\newcommand{\checkmerge}{\sc CheckMerge}
\newcommand{\checkdown}{\sc CheckDown}
\newcommand{\checkup}{\sc CheckUp}
\newcommand{\checkdef}{\sc CheckDefault}

\newcommand{\msgmerge}{\sc MsgMerge}
\newcommand{\msgdown}{\sc MsgDown}
\newcommand{\msgup}{\sc MsgUp}
\newcommand{\msgdef}{\sc MsgDefault}

\newcommand{\Bt}{B\mbox{-}tree}

%\newtheorem{theorem}{Theorem}
%% \newtheorem{hypothese}{Hypothèse}
%% \newtheorem{lemma}{Lemma}
%% \newtheorem{corollary}{Corollary}
%% \newtheorem{proposition}{Proposition}
%% \newtheorem{definition}{Definition}
%% \newtheorem{assumption}{Assumption}
%\newtheorem{remark}{Remark}

\floatname{algorithm}{Algorithm}

\newcommand{\phyreq}{{\tt phyReq}}
\newcommand{\phyreqinitiator}{{\tt phyReqInitiator}}
\newcommand{\updatesuccessor}{{\tt updateSuccessor}}
\newcommand{\host}{{\bf INSERT}}

\floatname{algorithm}{Algorithm}

\makeatletter
\providecommand*{\toclevel@algorithm}{0}
\makeatother
 
\title{CarPal: interconnecting overlay networks for a community-driven
  shared mobility\thanks{Supported by AEOLUS FP6-IST-15964-FET
    Proactive.}}
%
\author{Vincenzo Ciancaglini \and Luigi Liquori \and Laurent Vanni}

\institute{INRIA Sophia Antipolis M\'editerran\'ee, France\\
  Email: {\tt firstName.lastName@sophia.inria.fr} }

\pagestyle{plain}
\sloppy

\begin{document}

  
\maketitle
  
\begin{abstract} 
  Car sharing and car pooling have proven to be an effective solution
  to reduce the amount of running vehicles by increasing the number of
  passengers per car amongst medium/big communities like schools or
  enterprises.  However, the success of such practice relies on the
  community ability to effectively share and retrieve
  information about travelers and itineraries.  Structured overlay
  networks such as Chord have emerged recently as a flexible solution
  to handle large amount of data without the use of high-end servers,
  in a decentralized manner.  In this paper we present CarPal, a
  proof-of-concept for a mobility sharing application that leverages a
  Distributed Hash Table to allow a community of people to
  spontaneously share trip information without the costs of a
  centralized structure.  The peer-to-peer architecture allows
  moreover the deployment on portable devices and opens new scenarios
  where trips and sharing requests can be updated in real time.  Using
  an original protocol already developed that allows to
  interconnect different overlays/communities, the success rate
  (number of shared rides) can be boosted up thus increasing the
  effectiveness of our solution. Simulations results are shown to give
  a possible estimate of such effectiveness.
 
  \smallskip {\bf Keywords.} Peer to peer, overlay networks, case
  study, information retrieval, car sharing.
\end{abstract}





\section{Introduction\label{sec:introduction}}

%
%
%
\subsection{Context}
%
Car pooling is the shared use of a driver's personal car with one or
more passengers, usually but not exclusively colleagues or friends,
for commuting (usually small-medium recurring trips, like \eg\ home to
work, home to school, you name it). Amongst the many advantages it
decreases traffic congestion and pollution, reduces trip expenses
by alternating the use of the personal vehicle amongst different
drivers and enables the use of dedicated lanes or reserved parking
places where made available by countries aiming to
reduce the global dependency of petrol.

Car sharing is a model of car rental for short period of time (in
opposite of the classical car rental companies), where a number of
cars, often small and energy-efficient, are spread on a small
territory, like a city. Customers first subscribe to a company who
exploits and maintains the car park, then use those cars for their
personal purposes. Service fees are normally per kilometer and
insurance and fuel costs are included in the rates. Car sharing is an
interesting option for families that need a second car but do not
want to buy it. Modern geolocation technologies using GPS and mobile
phones help to find the closest car to pick. The same
economical/ecological advantage of car pooling applies, and
mathematically speaking they are parameters of the same function we
want to minimize.

\subsection{Problem overview}
%
In Car* services, an Information System (IS) has been showed to be
essential to match the offers, the requests, and the resources. The
Information System is, in most cases, a front-end web site connected
with a back-end database.  A classical client-server architecture is
usually sufficient to manage those services. Users register their
profile to one Information System and then post they
offers/requests. In presence of multiple Car* services, for technical
and/or commercial reasons, it is not possible to share contents across
the different providers, despite the evident advantage. As a simple
example the reader can have a quick look on those two web sites
Equipage06\footnote{\url{http://www.covoiturage-cg06.fr/}.} and Otto
and co\footnote{\url{http://www.ottoetco.org/}.} concerning car
pooling in the French Riviera region. At the moment those two web
sites does not communicate at all and do not share any user profile
neither they share any request, even if they operate on the same
territory and with the same objectives. Since both services are no
profit, the reason for this has probably to be found in the
client-server nature of both Information Systems that, by definition,
are not incline to collaborate with each other. Although in principle
this does not affect the correct behavior of both services, it is
clear that \emph{``In union is strength''}\footnote{Italian
  proverb.}. Moreover, the classical shortcomings of client-server
architectures make both service unavailable in case both servers are
down.


\subsection{Contributions}
%
As main contributions in this paper:
%
\begin{itemize}
\item we design and implement a Peer-to-peer based Carpool information
  system, that we call \emph{CarPal}: this service is suitable to be
  deployed with a very low infrastructure and can run on various
  devices spanning from PC to a small intelligent devices, like
  smartphones;

\item we customize the Arigatoni protocol and his evolution, the
  Synapse protocol, both specialized for resource discovery in overlay
  networks in order to allow two completely independent CarPal-based
  Information systems to communicate without the need of merging one
  CarPal system into the other or, even worse, build a third CarPal
  system including both.
\end{itemize}



\subsection{Outline}
%
The rest of the paper is organized as follows: in
Section~\ref{sec:link} we describe the interconnection of different
CarPal systems by means of the Synapse protocol developed in our
team. In Section~\ref{sec:architecture}, we introduce our CarPal
service and we show how it is mapped onto a Distributed Hash Table.
In Section~\ref{sec:proof} we show a running example with a
proof-of-concept that we have implemented in our team on the basis of
a real case of study in our French Riviera area of Sophia Antipolis a
technological pole of companies and research centers. A GUI is also
presented\footnote{See~\url{http://www-sop.inria.fr/teams/lognet/carpal}.}.
Section~\ref{sec:experiences} describes the deployment of a client
prototype\footnote{See~\url{http://www-sop.inria.fr/teams/lognet/synapse}.}
over the Grid'5000 platform.  Section~\ref{sec:conclusion} concludes
and presents some further work.


\section{Interconnecting multiple networks\label{sec:link}}

\subsection{Context and Motivations}
%
The choice of having a CarPal service per community (companies,
universities, etc.) comes from the need of having groups of people with
common travel patterns, habits, activities or even interests.  However
it's easy to see how such approach leads to the situation where two or
more different communities might be \emph{geographically overlapping},
\ie\ residing in the same area and \emph{not be aware of each other}
thus one not taking advantage of the other's free places. Often
companies are very close geographically and they have the same working
timetable. If, as intended, different companies have put in place
different CarPal communities, those possible matches will not be
taken into account.

As said in the introduction, any attempt of make different and
disconnected institutional, client-server based, Carpool services does
not foresee any form of service interconnection with the unpleasant
consequence of loosing potential matches between offers and requests
between users of different communities but living in the same region.
Very often the problem is not (only) technological: institutions do
not want to share their databases, or we don't trust the reputation
of other companies, or simply we want to travel with people of the
same enterprise.

In order to circumvent such limitation, communities can be
interconnected using our Arigatoni~\cite{CCL08} and
Synapse~\cite{LTVBCM09,LTB09} protocols developed in our team during
the Aeolus project.  Such meta-protocol allows a request to be routed
through multiple overlays, even using different routing algorithms,
thus increasing the success rate of every request. In case of
Arigatoni, communications and routing inter-overlays goes through a
broker-2-broker negotiation, where a broker is a special peer that can
be considered as the leader of the overlay \cite{LC07b}. In case of
Synapse, that represents an evolution of Arigatoni taking advantage of
the DHT technology of structured overlay networks, crossing overlays
is achieved through co-located nodes, represented by peers who are, by
user's choice, member of several communities.  Such nodes are able
themselves not only to query multiple communities in order to find a
match but also to replicate requests passing through them from one
network to another and to collect the multiple results. In the next
subsection we will briefly introduce the synapse protocol.


\subsection{Synapses in a nutshell}
%
The interconnection of overlay networks has been recently identified
as a promising model to cope with today's Internet issues such as
scalability, resource discovery, failure recovery or routing
efficiency, in particular in the context of information retrieval.
Some recent researches have focused on the design of mechanisms for
building bridges between heterogeneous overlay networks with the
purpose of improving cooperation between networks that have different
routing mechanisms, logical topologies and maintenance
policies. However we are still missing  more comprehensive approaches of such
interconnections for information retrieval and both quantitative and
experimental studies of its key metrics, such as satisfaction rate or
routing length.

Many disparate overlay networks may not only simultaneously co-exist
in the Internet but also compete for the same resources on shared
nodes and underlying network links. One of the problems of the overlay
networking area is how heterogeneous overlay networks may
\emph{interact} and \emph{cooperate} with each other. Overlay networks
are heterogeneous and basically unable to cooperate each other in an
effortless way, without merging, an operation which is very costly
since it not scalable and not suitable in many cases for security
reasons. However, in many situations, distinct overlay networks could
take advantage of cooperating for many purposes: collective
performance enhancement, larger shared information, better resistance
to loss of connectivity (network partitions), improved routing
performance in terms of delay, throughput and packets loss, by, for
instance, cooperative forwarding of flows.

As a basic example, let us consider two distant databases. One node of
the first database stores one $(key, value)$ pair which is searched by
a node of the second one. Without network cooperation those two nodes
will never communicate together. As another example, we have an
overlay network where a number of nodes got isolated by an overlay
failure, leading to a partition: if some or all of those nodes can be
reached via an alternative overlay, than the partition could be
recovered via an alternative routing.

In the context of large scale information retrieval, several overlays
may want to offer an aggregation of their information/data to their
potential common users without losing control of it.  Imagine two
companies wishing to share or aggregate information contained in their
distributed databases, obviously while keeping their proprietary
routing and their exclusive right to update it.  Finally, in terms of
fault-tolerance, cooperation can increase the availability of the
system, if one overlay becomes unavailable\,the global network will
only undergo partial failure as other distinct resources will be
usable.

We consider the tradeoff of having one \vs\ many overlays as a
conflict without a cause: having a single global overlay has many
obvious advantages and is the \textit{de facto} most natural solution,
but it appears unrealistic in the actual setting. In some optimistic
case, different overlays are suitable for collaboration by opening
their proprietary protocols in order to build an open standard; in
many other pessimistic cases, this opening is simply unrealistic for
many different reasons (backward compatibility, security, commercial,
practical, etc.). As such, studying protocols to interconnect
collaborative (or competitive) overlay networks is an interesting
research vein.

The main contribution of this research vein is to introduce, simulate
and experiment with \emph{Synapse} \cite{LTVBCM09}, a scalable
protocol for information retrieval over the inter-connection of
heterogeneous overlay networks. The protocol is based on co-located
nodes, also called \emph{synapses}, serving as low-cost natural
candidates for inter-overlay bridges. In the simplest case (where
overlays to be interconnected are ready to adapt their protocols to
the requirements of interconnection), every message received by a
co-located node can be forwarded to other overlays the node belongs
to. In other words, upon receipt of a search query, in addition to its
forwarding to the next hop in the current overlay (according to their
routing policy), the node can possibly start a new search, according
to some given strategy, in some or all other overlay networks it
belongs to. This obviously implies to providing a Time-To-Live value
and detection of already processed queries, to avoid infinite loop in
the networks, as in unstructured peer-to-peer systems.  Applications
of top of Synapse see those inter-overlay as a unique overlay.

We also study interconnection policies as the explicit possibility to
rely on \emph{social} based strategies to build these bridges between
distinct overlays; nodes can invite or can be invited. In case of
concurrent overlay networks, inter-overlay routing becomes harder, as
intra-overlays are provided as some black boxes: a \emph{control}
overlay-network made of co-located nodes maps one hashed key from one
overlay into the original key that, in turn, will be hashed and routed
in other overlays in which the co-located node belongs to. This extra
structure is unavoidable to route queries along closed overlays and to
prevent routing loops.

Our experiments and simulations show that a small number of
well-connected synapses is sufficient in order to achieve almost
exhaustive searches in a ``synapsed'' network of structured overlay
networks. We believe that Synapse can give an answer to circumventing
network partitions; the key points being that:
  %
   \begin{itemize}
   \item several logical links for one node leads to as many
     alternative physical routes through these overlay, and
  %
   \item a synapse can retrieve keys from overlays that it does not
     even know simply by forwarding their query to another synapse
     that, in turn, is better connected.
 \end{itemize}
%
 Those features are achieved in Synapse at the cost of some additional
 data structures and in an orthogonal way to ordinary techniques of
 caching and replication. Moreover, being a synapse can allow for the
 retrieval of extra information from many other overlays even if we
 are not connected with.  Finally, Synapse can either work with
 ``open'' overlays adapting their protocol to synapse interconnection
 requirements, or with ``closed'' overlays that will not accept any
 change to their protocol. Figure \ref{fig:example} shows how Synapse
 can give an answer in case of routing across differents
 intra-overlays and dealing with network partitions. For more details,
 see \cite{LTVBCM09}.



\begin{figure}[!t]
  \centering
  \mbox{\includegraphics[width=0.4\columnwidth]{fig/GET_into_other_ON.pdf}\,\,
    \includegraphics[width=0.4\columnwidth]{fig/ON_with_network_partition.pdf}}
  \caption{Routing across differents overlays and dealing with a network partition\label{fig:example}}
\end{figure}

\section{Application architecture \label{sec:architecture}}

\subsection{Application principles}
%
One of the most important features for a car share application is to
be able to maximize the chances of finding a match between one driver
and one or more travelers. From this comes the choice of arranging
the database by communities in order to put in touch people who most
likely share the same traveling patterns in space and time (\eg\ work
for the same company, attend the same university and so on). Another
important aspect is to be able to update the planned itinerary
information as quick as possible so that a last minute change in plans
can be easily managed and updated and may eventually lead in finding a
new match.
 
For the above reasons, CarPal has been intended as a desktop and
mobile application running on a peer-to-peer overlay network. This
allows a community of people to spontaneously create its own travel DB
(which, as it will be shown later, can be interconnected with siblings
communities) and manage it in a distributed manner.  Moreover, it
constitutes a flexible infrastructure where,by being deployed on
connected mobile devices, it will be possible to develop more advanced
info-mobility solutions that might take into account the position of
the user/vehicle (via the internal GPS), geographically-aware network
discovery or easy network join or vehicle tracking through checkpoints
with the use of Near Field Communications technologies~\cite{NFC}.

\subsection{CarPal in a nutshell}
%
The working principle is simple: a user running a CarPal client on his
mobile or desktop will connect to one or more communities to which he
is member (\ie\ he has been invited or his request has been accepted).
Tow operations are then available, namely $(i)$ publishing a new
itinerary and $(ii)$ finding a matching itinerary.

\noindent{\bf Publishing a new itinerary.}
%
When a CarPal user has a one time or recurring trip that he wants to
optimize cost-wise he can publish his route in the community in hope
to find someone looking for a place in the same route and time window
to share the ride with. A planned itinerary is usually composed by the
following data:
%
\begin{itemize}
\item \emph{Trip date and number of repetitions} in case is a
  recurring trip;
\item \emph{Departure place and arrival place}, whose representation
  is critical since a too high granularity might lead to miss similar
  results;
\item \emph{Departure time} or at least an estimate given by the
  user;
\item \emph{Arrival time} or at least an estimate given by the user;
\item \emph{Number of available seats} to be updated when another
  passenger asks for a place;
\item \emph{Contact}, usually an e-mail or a telephone number;
\item \emph{Information}, other useful information, \ie\ animal
  allergies, women-only car etc.
\end{itemize}
%
Moreover, from a functional point of view, a trip \eg\ from a place A
to a place D may include several checkpoints, meaning that the user
offering his car can specify one or more intermediate places where he
is willing to pick up or leave a passenger.

 Once the user has inserted all the needed data (date, departure,
arrival, times, seats and optional checkpoints), the trip is 
decomposed in all the possible combinations: for example, a trip 
containing the legs A-B-C-D (where B and C are checkpoints specified 
by the user) will generate the combinations A-B and A-C and  
A-D and  B-C and  B-D and  C-D. This
operation is commonly known as \emph{Slice and Dice}.  Since the
number of possible combinations can increase exponentially with the
number of checkpoints, there is a software limitation to 3 maximum
stops in the trip. Each combination is then stored in the DHT as an
individual segment; moreover all the segments who don't start form A
are marked as estimated in departure time since, given a trip made of
different checkpoints, only the effective departure time can be
considered reliable, the others being subject to traffic conditions
and contingencies. Geographic and time information must be encoded in
such a way to be precise enough to be still relevant for our purposes
(someone leaving from the same city but 10 km far is not a useful
match) yet broad in a way that a punctual query will not skip
important results.

Until we find a feasible way to perform semantically relevant range
queries, departure/arrival hotspots can be added in a ``social''
style. A driver departing from a certain spot for the first time will
be offered a map where to mark his spot of departure, inviting the
user to use an already defined spot nearby in case. Finally,
concerning time approximation, a 20 minutes window is used to
approximate departure times. Both during an insertion or a query,
anything within the 0-19 minutes interval will be automatically set at
10 minutes, 20-39 will be set at 30 minutes and 40-59 at 50.

\noindent{\bf Finding a matching itinerary and one seat.}
%
A user wishing to find a ride can perform a search by inserting the
following information:
%
\begin{itemize}
\item \emph{Date} of the trip;
\item \emph{Departure} place and time (picked on a map between the
  proposed points;
\item \emph{Arrival} place and wished time, picked in the same
  manner as the departure.
\end{itemize}
%
To increase the chances of finding a match, only part of the search
criteria can be specified, allowing \eg\ to browse for all the trip
leading to the airport in a certain day disregarding the departure
time (giving the user the chance of finding someone leaving the hour
before) or the departure point (giving the user, in case of nobody
leaving from the same place as him, to find someone leaving nearby to
join with other means of transportation).  Moreover it is possible to
specify checkpoints in the search criteria too, in order to have the
system look for multiple segments and create aggregated responses out
of publications from multiple users.

\noindent{\bf Negotiation.}
%
Once the itinerary has been found it will be possible to contact the
driver in order to negotiate and reserve a seat. If the trip is an
aggregation of different drivers' segments all of them will be
notified through the application. The individual trip records will
then be updated by decreasing the available seats number.

 
\begin{table}[!t]
  {\small \begin{center}
    \key{
      \begin{tabular}{|c|c|c|c|}
        \hline
        \key{\bf Criteria} &\key{\bf Key} & \key{\bf Value} & \key{\bf Trip Grouped by} \\
        \hline
        1 & TR|TRIP\_ID & $\clubsuit$ & Individual \\\hline
        2 & T|DATE|DEP|TOD|ARR|TOA &list[TRIP\_ID] 
        & Dep \& Arr \& Time\\ \hline   
        3 & DA|DATE|DEP|ARR & list[TRIP\_ID] 
        & Dep \& Arr\\ \hline
        4 & D|DATE|DEP & list[TRIP\_ID] 
        & Dep\\ \hline
        5 & A|DATE|ARR & list[TRIP\_ID] 
        & Arr\\ \hline
        6 & U|USER\_ID & list[TRIP\_ID] 
        & User id\\ \hline
      \end{tabular}

     \smallskip where $\clubsuit$ = \key{[DATE,DEP,TOD,ARR,TOA,SEATS,REF,PUB]}
    }
  \end{center}}
  \caption{Different data structures stored in the DHT for each entry}
  \label{t:DHTvalues}
\end{table}

\subsection{Encoding CarPal in a DHT }
%
The segments are stored in the DHT according to Table
\ref{t:DHTvalues}.
%
Since we are not able yet to perform useful range queries on the DHT,
multiple keys are inserted or updated for each entry, representing
sets of trips grouped according to different criteria:
%
\begin{enumerate}
\item Is the actual trip record, associated to a unique
  \key{TRIP\_ID}, that will be updated, for example, when someone book
  a seat.  The information stored concerns trip date \key{DATE},
  departure place \key{DEP} and time \key{TOD}, arrival place
  \key{ARR} and time \key{TOA} number of available seats \key{SEATS}
  (or cargo space, in case of shared goods transportation), a
  reference to contact the driver \key{REF}, and if the trip has to
  be public or not \key{PUB}.  Depending on the needs more information
  can be appended to this record;
\item Represents a set of trips having the same date,
  departure/arrival point and departure/arrival time. The key is
  created by concatenating the token \key{T}, trip date \key{DATE},
  departure place \key{DEP}, departure time \key{TOD}, arrival place
  \key{ARR} and arrival time \key{TOA}.  As value is the list of
  \key{TRIP\_ID} pointing to the trip records. The key is created by
  appending the token \key{TR} to the \key{TRIP\_ID};
\item Is a set of trips grouped by date, departure and arrival place.
  It will be used to query in one request all the trips of the day on
  a certain itinerary. The key to store them in the DHT is
  consequently made by appending the token \key{DA}, trip date,
  departure and arrival point;
\item[4-5.] Are two sets of trips arranged by day and by departure
  or arrival point. The key is therefore made by concatenating either
  the token \key{D} (for departure) or \key{A} (for arrival) to the
  \key{DATE} and departure or arrival point \key{DEP} or \key{ARR}.
  This key can be used \eg to query in one request all the trips of
  the day leaving from a company or all the trips of the day heading
  to the airport; \setcounter{enumi}{5}
\item Is a set of trips for a given user. The key is therefore the
  token \key{U} prepending the \key{USER\_ID} itself.
\end{enumerate}

\subsection{Network architecture}
%
The overlay chosen for the proof of concept is Chord~\cite{chord}
although other protocol could be used to leverage the locality of the
application or a more direct geographical mapping (see Section
\ref{sec:semantic}).  Even on a simple Chord mechanisms to ensure
fault tolerance can be put in place, like data replication using
multiple hash keys or request caching.  To allow a new community to be
start up, a \emph{public tracker} has been put in place on the
Internet. The public tracker is a server whose tasks can be resumed as
following:
%
\begin{itemize}
\item It allows the setup of a new community by registering the IP of
  certain reliable peers, in a YOID-like
  fashion~\cite{francis2000yoid};
\item It acts as a central database of all the communities, keeping
  track of them and their geographical position;
\item consequently, it can propose nearby overlays to improve the
  matches by placing co-located peers;
\item It acts as a third party for the invitation of new peers into an
  overlay;
\item It can provide statistical data about the activity of n overlay,
  letting a user know if a certain community has been active lately
  (and thus it's worth joining);
\item It stores all the placeholders set as departure and arrival
  points, in order to avoid having similar routes not matching because
  of 2 different geographical markers for the same spot;
\item It act as the entry point to download the application and get
  updates.
\end{itemize}

\subsection{Enhancing the architecture}
%
It is clear that a user might take advantage of nearby communities
other than his. In case of an unsuccessful query or upon explicit
request, it is possible to access nearby networks by asking co-located
nodes in his community to reroute the query.  To do this, synapse
nodes are connected, in parallel to their actual overlays, to a common
Control Network that handles 2 different data structures (a KeyTable
and a CacheTable) used to manage inter-overlay routing. Both are
implemented as DHTs on a global overlay participated by every node of
every networks.

\noindent{\bf The Key Table}
%
is responsible for storing the unhashed keys circulating in the
underlying overlays.  When a synapse performs a \key{GET} that has to
be replicated in other networks, it makes the unhashed key available
to the other synapses through the Key Table. The key is stored using
an index formed by a networks identifier as a prefix and the hashed
key itself as a suffix. This way when a synapse on the overlay with
\eg\ \key{ID = A} will have to replicate \eg\ \key{H(KEY) = 123}, it
will be able to retrieve, if available, the unhashed key from the
KeyTable by performing a get of \key{A|123}.

\noindent{\bf The Cache Table}
%
is used to implement the replication of get requests, cache multiple
responses and control the flooding of foreign networks.  It stores
entries in the form of \key{[H(KEY),TTL,[NETID],[CACHE]]}.  In a
nutshell: \key{NETID} are optional and used to perform selective
flooding on specific networks.  When another synapse receives a
\key{GET} requests, it checks if there is an entry in the Key Table
(to retrieve the unencrypted key), and an entry in the Cache Table; if
so, it replicates the \key{GET} in the \key{[NETID]} networks he is
connected to, or in all his networks if no \key{[NETID]} are
specified. All the responses are stored in the \key{[CACHE]} and only
one is forwarded back, in order not to flood other nodes having
performed the same request. A \key{TTL} is specified to manage the
cache expiration and block the flooding of networks.  When the synapse
originating the request receives the first response, it can retrieve
from the Cache Table the rest of the results.  The cached responses
should be sent back with the associated \key{NETID}. This can allow a
with time to define a strategy of selective flooding to the networks
who are better responding to a synapse request.

\noindent{\bf Inter-overlay routing algorithm} is performed hen a peer
wish to perform query: before routing the request in his own community
it adds an entry in the KeyTable, containing the unhashed key to be
searched, and an empty entry in the CacheTable.  When an synapse in
the first overlay receives the request, it looks for the unhashed key
in the KeyTable and the corresponding entry in the CacheTable. If
those are found, the co-located synapse will query for the same key in
all his communities and store the results in the CacheTable, in order
not to pollute the network with too many results. The requesting peer
in the first network will then collect the results from the
CacheTable.

\noindent{\bf Controlling the data.}
%
Since different CarPal overlays use different hash functions to map
their keys a first level of privacy and control is guaranteed in case
a community wish to have some control over the visibility of their
information. At the present state there are 2 possible scenarios for
accessing the data:
%
\begin{itemize}
\item A user can search for trips mark as both public and private in
  every overlay he's directly connected to. As previously stated, the
  connection to an overlay happens via invitation through a mechanism
  similar to certain social networks;
\item If certain nodes of his own networks are member of other
  overlays, they can act as synapses and route queries from one
  network to another. However only the trips marked as ``public'' will
  be made available to a foreign request.
\end{itemize}


\section{A Running example \label{sec:proof}}

We hereby present a first proof-of-concept for a CarPal application
implementing the concepts exposed above.  The software is still at an
initial development but it has already been proven to be working in
posting new routes and querying them across multiple networks.  A
basic user interface is proposed showing a first attempt to integrate
a mapping service (namely, Google Maps \cite{googlemaps:website}) in
the application, to render the user experience more pleasant and
efficient.
 
%
\subsection{Building the scenario}
%
Let's see a practical example to better explain the logic behind the
application.  As a real world scenario for our proof-of-concept we
chose the area area of Sophia Antipolis in the department of
Provence-Alpes-Cote d'Azur, France. The area (Figure \ref{tab:journey}
left) constitutes an ideal study case, being a technological pole with
a high concentration of IT industries and research centers, thus
providing several potential communities of people working in the same
area and living in nearby towns (Antibes, Nice, Cagnes sur Mer to name
some).
 

An engineer working in the area and willing to do some car pooling in
order to reduce his daily transfer costs can publish his usual route
to the CarPal overlay specific to his company. We assume the network
has been already put in place spontaneously by him or some colleague
of his.  He can then use the CarPal application to publish his route
with an intermediate checkpoint (as shown in Figure
\ref{fig:publish1}).
%
\begin{figure}
  \begin{center}
    \begin{tabular}{c@{\hspace{2mm}}c}
      \includegraphics[scale=0.38]{fig/screenshot/geolock2}
      &
      \begin{tabular}[b]{l}
        \begin{tabular}{|l|l|} \hline 
          Trip date & 15/01/2010 \\ \hline 
          Departure & Nice \\ \hline 
          Departure Time & 8.00 \\ \hline
          Checkpoint & Cagnes sur Mer \\ \hline 
          Checkpoint Time & 8.30 \\ \hline
          Arrival & Sophia Antipolis \\ \hline 
          Arrival Time & 9.00 \\ \hline
          Seats available & 4 \\ \hline 
          Contact & jsmith@email.com \\ \hline
        \end{tabular}
        \\\\
        \begin{tabular}[b]{|l|l|} \hline 
          Nice-Sophia & 8.00-9.00 \\ \hline
          Nice-Cagnes sur Mer & 8.00-8.30 \\ \hline
          Cagnes sur Mer-Sophia & 8.30-9.00 \\ \hline
        \end{tabular}
      \end{tabular}
    \end{tabular}
  \end{center}
  \caption{The ``Geo'' set-up and the journey data (right) and sliced / diced segments (bottom right)}
  \label{tab:journey}
\end{figure}
%
\begin{figure}[!t]
\begin{center}
\includegraphics[scale=0.4]{fig/screenshot/enterprisePub}
\end{center}
\caption{CarPal application publishing a new trip}
\label{fig:publish1}
\end{figure}
%
As previously described, there is a checkpoint where our user
is willing to stop and pick up some passengers.

\subsection{Slice and Dice and encoding in the DHT}
%
Starting from the above data all the possible combinations are
generated leading to the segments shown in Figure \ref{tab:journey}
(right). Only the differences are reported, each of those segments
share the same date, available seats and contact information.  The 3
segments are then stored in the DHT by updating the appropriate keys
or adding new ones, as shown in Table \ref{tab:keys}.  Time and dates
are converted to appropriate strings while geographical positions are
identified by a placeholder (\ie\ \key{NICE}, \key{SOPH}...)
representing a record accessible in the DHT or the central tracker.
%
\begin{table}
  \begin{center}
    \key{\small{
        \begin{tabular}{|c|l|l|c|}
          \hline
          \bf{Criteria} & \multirow{2}{*}{\bf{Operation}} & \multirow{2}{*}{\bf {Key}} & \multirow{2}{*}{\bf {Value}} \\ 
          \bf{(see Table \ref{t:DHTvalues})} & & & \\ \hline
          1 & PUT & TR 123 & $\clubsuit$ \\ \hline
          1 & PUT & TR 124 & $\spadesuit$ \\ \hline
          1 & PUT & TR 125 & $\blacksquare$ \\ \hline
          2 & APPEND & T 20100115 NICE 0800 SOPH 0900 & 123 \\ \hline
          2 & APPEND & T 20100115 NICE 0800 CAGN 0830 & 124 \\ \hline
          2 & APPEND & T 20100115 CAGN 0830 SOPH 0900 & 125 \\ \hline
          3 & APPEND & DA 20100115 NICE SOPH & 123 \\ \hline
          3 & APPEND & DA 20100115 NICE CAGN & 124 \\ \hline
          3 & APPEND & DA 20100115 CAGN SOPH & 125 \\ \hline
          4 & APPEND & D 20100115 NICE & 123 \\ \hline
          4 & APPEND & D 20100115 NICE & 124 \\ \hline
          4 & APPEND & D 20100115 CAGN & 125 \\ \hline
          5 & APPEND & A 20100115 SOPH & 123 \\ \hline
          5 & APPEND & A 20100115 CAGN & 124 \\ \hline
          5 & APPEND & A 20100115 SOPH & 125 \\ \hline
          6 & APPEND & U jsmith@email.com & [123,124,125] \\ \hline
        \end{tabular}
      }}
  \end{center}
  
where $\clubsuit$ = [20100115, NICE, 0800, SOPH,0900, 3, jsmith@email.com, public=true] \\
where $\spadesuit$ = [20100115, NICE, 0800, CAGN,0830,3, jsmith@email.com, public=true] \\
where $\blacksquare$ = [20100115, CAGN, 0830, SOPH, 0900, 3, jsmith@email.com, public=true] \\
\caption{DHT operations}
  \label{tab:keys}
\end{table}

A PUT operation represents the insertion of a new key not yet existing
whereas the APPEND operation assumes that the key might be already in
the DHT, in which case the value is simply updated by adding the new
entry to the list.
After the insertion the trip is published and available to be
searched. From Figure \ref{fig:publish1} we can see that it's possible
to set as an option that the trip will stay private.  In that case,
the segments will be discoverable only by peers members of the same
network.

\subsection{Searching for a trip}
%
Trip searching happens in a similar way as the trip submission. As we
can see in Figure \ref{fig:searchAggregate} (left) the user can
specify an itinerary, a specific time and even some intermediate
segments to find all the possible combinations. Depending on the
search criteria specified, the application will perform a query for
either a key made of Time of Departure and Time of Arrival, for a more
exact match, a key with only Departure and Arrival to browse through
the day's trips or a key with only Departure or Arrival point for a
broader search.  Thanks to the Slice and Dice operation it is possible
to aggregate segments coming from different users as Figure
\ref{fig:searchAggregate} (right) shows.

\begin{figure}[!t]
\begin{center}
\mbox{\rew{10}\includegraphics[scale=0.3]{fig/screenshot/NiceSophiaSub}
~\includegraphics[scale=0.3]{fig/screenshot/NiceCagnesAntibesSub}}
\end{center}
\caption{Simple search \vs\ aggregate results}
\label{fig:searchAggregate}
\end{figure}

This way the driver has more possibilities to find guests in his
car. Despite that there can still be some places available for his
daily route. To optimize even further he might share his information
with, for example, students of nearby universities with their own
carpool network (that has the same functions and technology).

By marking his published itinerary as public, a member of the
Enterprise Network allow the students to get matching results via a
synapse (Figure \ref{fig:searchForeign}), \ie\ somebody registered to
both networks (Figure \ref{fig:studententerprise}).  This allows the
system to increase the chances of finding an appropriate match while
maintaining good locality properties.

\begin{figure}[!t]
\begin{center}
\mbox{\rew{15}\includegraphics[scale=0.3]{fig/screenshot/EnterpriseJoinStudent}~
\includegraphics[scale=0.31]{fig/screenshot/NiceCagnesAntibesSophiaSub_StudentNetwork}}
\end{center}
\caption{Synapse creation (left). CarPal Students accessing result from Enterprise Network (Right)}
\label{fig:searchForeign}
\end{figure}

 
\begin{figure}[!t]
\begin{center}
\includegraphics[scale=0.3]{fig/screenshot/Student-Enterprise}
\\[-10mm]
\includegraphics[scale=0.3]{fig/screenshot/Synapse-Network}
\end{center}
\caption{Students, Enterprise and Synapsed Overlay Networks}
\label{fig:studententerprise}
\end{figure}

 


\section{Experimental results\label{sec:experiences}}

\subsection{Deployment settings}
%
In order to test our inter-overlay protocol on real platforms, we have
initially developed JSynapse, a Java prototype that fully
implements a Chord-based inter-overlay network.  We have experimented
with JSynapse on the Grid'5000 platform connecting more than $20$
clusters on $9$ different sites. Again, Chord was used as the
intra-overlay protocol.  The created Synapse network was first made of
up to $50$ processors uniformly distributed among $3$ Chord
intra-overlays. Then, still on the same cluster, as nodes are
quad-core, we deployed up to $3$ logical nodes by processor, thus
creating a $150$ nodes overlay network, nodes being dispatched
uniformly over $6$ overlays. During the deployment, overlays were
progressively bridged by synapses (the degree of which was always
$2$).

\subsection{Experiences results}
%
Figure~\ref{dep:1-sat} (left) shows the satisfaction ratio when
increasing the number of synapses (for both white and black box
versions). A quasi-exhaustiveness is achieved, with only
a connectivity of $2$ for synapses.
Figure~\ref{dep:1-sat} (right) illustrates the very low latency (a few
milliseconds) experienced by the user when launching a request, even
when a lot of synapses may generate a lot of messages. Obviously, this
result has to be considered while keeping the performances of the
underlying hardware and network used in mind. However, this suggests
the viability of our protocols,\,the\,confirmation\,of\,simulation
results,\,and\,the\,efficiency\,of\,the\,software\,developed.
%
\begin{figure}[!t]
  \includegraphics[width=0.5\linewidth]{fig/dep1-sat.pdf}
  \includegraphics[width=0.5\linewidth]{fig/dep1-time.pdf}
  \up{4}
  \caption{Deploying Synapse : Exhaustiveness and Latency \label{dep:1-sat}}
  \up{4}
\end{figure}

\subsection{Results interpretation}
%
To understand such results, let's consider the following proof: We
call $R$ a region where different peer populations, $P$ and $Q$,
coexist.  At a first stage the two populations have no interactions,
\ie\ $P \cap Q = \emptyset$.  We call $\{ Pub_{P} \}$ and $\{ Sub_{P}
\}$ the number of publications (available trips) and searches
(passengers) in the community/overlay $P$, whereas $\{ Pub_{Q} \}$ and
$\{ Sub_{Q} \}$ are the publications and subscriptions for the
community $Q$.  In a non-interconnected environment, being $\{
Match_{P} \}$ and $\{ Match_{Q} \}$ the number of matches between a
driver and a passenger, an indication of the success of this solution
might be given by $SuccessRate = \# \{ Match_{P} \cup Match_{Q} \}$,
that is the number of matches within the single networks.  By
introducing Synapses to interconnect several communities we change the
assumptions to $P \cap Q \neq \emptyset$ and introduce a new
population $S \subset P \cap Q$, who represents peers residing in $R$
being connected to both $P$ and $Q$ overlays and a new match
figure, $$\{ Match_{P\cup Q} \}=\{ Pub_{P}\,match\,Sub_{Q} \}_{\forall
P,Q \notin S } \cup \{ Pub_{Q}\,match\,Sub_{P}\}_{\forall P,Q
\notin S}$$ which represents the number of matches between an offer
from a peer connected to only one community and a request from a peer
from another community, thanks to the new interconnection provided by
the synapses.  Our success rate becomes then
$$SuccessRate = \# \{ Match_{P} \cup Match_{Q} \cup Match_{P\cup Q} \}$$

The above results show the rate to which $Match_{P\cup Q}$ increases
depending on the number of synapses in an overlay.  Moreover here we
see that with a sufficient rate of co-located nodes the exhaustiveness
of multiple connected communities becomes comparable as if the peers
where all in the same overlay, with the main advantage of scalability
and data locality.  Imagining a worldwide service in fact, it is a big
advantage to handle only local data with the possibility though of
being able to query potentially every other overlay (\ie\ for
occasional long trips).
 


\section{Conclusion and further work\label{sec:conclusion}}

Among the potential improvement, we shortly mention a few ones.
%
\subsection{Improved network bootstrap and community discovery}
%
At the present state a new community can be setup or joint by passing
through the tracker. This keeps track of community activities, their
location, handles the join negotiation and restrictions and can
suggests nearby communities that could be joined, however it
constitutes also a centralized point of failure for all the
communities. To improve further more the mechanism the following
solutions can be put in place:
%
\begin{itemize}
\item Assuming that a community/overlay could very likely reside on
  the same network infrastructure (i.e. the enterprise intranet) a
  discovery protocol can be put in place leveraging existing
  technologies like Avahi \cite{avahi:website} to discover new peers
  or new networks to join;

\item Peer caching could be used to reconnect to previously connected
  peers whose activity is known to be reliable;

\item An invitation to a new community could be handled physically via
  an NFC transaction. A user with an NFC enabled phone could be
  invited by another user by simply swiping the phones
  together. Furthermore this could be an additional guarantee of a
  user ``reliability'' as the new participant needs to be known and
  met by an existing member;

\item The community database could be as well stored in the DHT
  itself, meaning by this that new communities could simply be
  discovered through specific requests routed through existing
  synapses to other networks in a ping-like way.
\end{itemize}


\subsection{Semantic queries and specialized protocols}\label{sec:semantic}
%
It appears clear that the current approach suffers from the limitation
of a simple key-value approach. Such approach does not fit well into
an application that finds her strength in the possibility of
performing searches according to many different criteria.  The
adoption of a semantic hash function (such as \cite{SemanticHash})
would allow to cluster in nearby peers information semantically close
(\ie\ trips heading to sibling destinations or taking place in the
same time window).  Needless to say, with such hashing in place the
adoption of an overlay protocol more suited to range queries (like
P-Ring \cite{P-Ring}, P-Grid\cite{P-Grid} or Skipnet
\cite{Harvey03skipnet:a}) might lead the semantically significant
range queries, where, for example, departure and arrivals can be
geographically mapped and queried with a certain range in Km.

Another possible improvement (currently under study) would be to use a
DHT protocol more suited for geo-located information. CAN \cite{CAN})
in a 2D configuration is a first example of how this could be
achieved. Mapping CAN's Cartesian space over a limited geographic area
(like in Placelab \cite{PlaceLab}) could ease the query routing and
eventually provide some strategic points to place synapsing nodes.

\subsection{Overlay-underlay mapping optimizations}
%
From a networking point of view, the scientific hard point to solve is
the overlay-underlay network mapping (to avoid critical latency issues
due to the fact that one ``logical'' hop may via ``many'' physical
hops); The issue is being investigated and could involve for example
the use of several always on-servers to be used to ``triangulate'' the
position of a peer over the internet (according to latency metrics)
and cluster together ``nearby'' peers (whereas nearby will mean sharing
similar latencies to the same given servers).  Another issue would be
to make the service firewall-resilent by implementing TCP Punch-hole
techniques in the peer engine and in the tracker.


\subsection{Backward compatibility with other Carpool services}
%
At the present time we have developed a CarPal service that is
suitable to interconnecting Carpool services whose standards are open,
collaborative and based on our software. Unfortunately this is not
always the case. This makes a backward compatibility problem in case
we want to connect existing services. To take into account this case,
the Synapse protocol we have developed allows also a, so called,
\emph{black box} variant, that is suitable to interconnecting overlays
that, for different reasons, are not collaborative at all, in the
sense that they only route packets according to their proprietary and
immutable protocol.  Being the black box more of a meta-protocol
running on top of existing, and not necessarily peer-to-peer,
protocols, we can imagine strategically placed Synapse nodes being
responsible of querying existing web services and returning the
corresponding information as if they were coming from a foreign
network. This open new scenarios were multimodality is easily
integrated and made available for the nearby communities. The system
needs to be properly designed in order to avoid a situation where too
many peers act as a Distributed Denial Of Service, but the current
infrastructure make it easily feasible.

\subsection{User rating, social feedback}
%
Being CarPal an application based on user-generated content and
designed to put in touch people not necessarily acquainted to each
other, it is important to implement some social feedback and security
mechanism to promote proactive and good behaviors by the users.
Borrowing from today's most successful web applications, 2 solutions
can be imagined:
%
\begin{itemize}
\item A user rating should be put in place in order, for example, to
  allow passengers to evaluate a driver's punctuality, behavior and
  driving skills, and vice-versa. This feedback, similar to what
  systems like Ebay already have since several years, can help
  maintaining a high level of quality in the service by giving an
  immediate picture of a driver or passenger's reliability;

\item Some points can be assigned to users based on their activity in
  the community. The more a user will be proactive by publishing or
  subscribing to new trips in an overlay, the more ``karma points'' he
  will receive.  A similar approach can be verified in Social News
  website like Digg~\cite{Digg} or Reddit~\cite{Reddit} and has become
  pretty common in today's social media.  With the deployment and
  integration of new distributed services, these points could act as a
  ``virtual cash'' and grant access to features normally reserved to
  paying customers, thus motivating drivers and passengers to keep a
  community alive.
\end{itemize}

\subsection{Other potential applications}
%
The Car sharing/pooling is not the exclusive applicative field the
overlay network technology we designed; with the same final objective
of minimizing traffic, pollution and energy a service interconnecting
transportation companies Information Systems could be envisaged. A
``BoxPal'' system could be easily build using the same overlay network
technology: the only difference being the (more difficult) 3D
bin-packing combinatorial algorithms employed instead of simple
matching of drivers/cars/itinerary/car places.

%%%%%
%%%%% SPARE MATERIAL  C O M M E N T !!!
%%%%%
\comment{ TUTTO RFID MATERIAL IN VRAC DA POMPARE....VEDI ANCHE NEL PDF
CHE TI HO MANDATO
%
Another potential application would be the introduction of the RFID
technologies that could be used to collect information about locations
and possessions of people (and hence about their preferences, possible
medical conditions, ...) without them knowing about it generated fears
in the general public.
%
\Gi{DS = discovery services}
%
The usability in the Enterprise Supply Chain is an immediate proof of
usage of our approach. From a networking point of view, the scientific
hard point to solve are the mapping overlay-underlay network (to avoid
critical latency issues due to the fact that one ``logical'' hop may
via ``many'' physical hops); - crossing firewalls; - merging different
overlay networks (\eg\ , networks of different companies); -
distributing semantic queries toward a P2P network; - trading off
between routing complexity and of routing method for exhaustive
discovery;- dealing with heterogeneous underlay networks (IP, Wifi,
Zigbee, GPRS), just to mention a few.
%
Our overlay technology has various originalities that render it very
interesting from a social viewpoint. First, its particular knowledge
sharing approach will permit databases and DSs to support much more
cooperation between people than currently. Second, it will also permit
people to design their own access control or usage rules and associate
them to the information they own. Third, thanks to our DSs, people (or
their personal agents) will better be able to find which databases
have stored information about them (then, if these persons are not
happy with that storage, they can legally ask a database
server/manager to remove or update such information). Fourth,
conversely, if these persons have their own personal databases where
they have described access control and usage rule for some kinds of
information about themselves (\eg, their age or their weight), thanks
to our DS, database servers could find such rules and respect
them. This last point is certainly a naive ideal but could be checked
by people thanks to the third point and then made well known by them
or rewarded in some other way.
%
Systems such as ``Google Latitude'' allows people to let persons they
trust track them. A flexible and secure DS could be uhypothesis.
%
 One objective of RFID3S is to build a P2P overlay network where peers
contain fragments of RFID databases. Due to the heterogeneity of the
network and due to the presence of peers playing different roles and
of SW and HW barriers between companies (very often competitors) we
plan to use a - so called - hybrid overlay networks. A popular
topology has a super-peer serving as a server for a subset of the
peers, as in well know Skype VOIP application.
%
Towards a discovery service in a scalable, secure and flexible
Internet of things including RFID and NFC devices.  The final
objective of RFID3S is to build a hybrid overlay network where
security, scalability and semantic expressivity of queries cohabit
without tears for companies and end-users. Using a P2P infrastructure
for modeling an ONS of RFID/NFC in presence of different kinds of
peers and different kind of underlay network is probably one of the
more interesting challenges of the project.  Routing also complex
queries towards this network in a secure way is also worth of
interest.
%
The usability in the Enterprise Supply Chain is an immediate proof of
usage of our approach. From a networking point of view, the scientific
hard point to solve are the mapping overlay-underlay network (to avoid
critical latency issues due to the fact that one ``logical'' hop may via
``many'' physical hops); - crossing firewalls; - merging different
overlay networks (e.g., networks of different companies); -
distributing semantic queries toward a P2P network; - trading off
between routing complexity and of routing method for exhaustive
discovery;- dealing with heterogeneous underlay networks (IP, Wifi,
Zigbee, GPRS), just to mention a few.  }



\comment {spostato tutto in fondo. Cominciamo dal tecnico e andiamo sul funzionale.}

\comment{Questo te l' ho commentato perche' o i concetti sono gia'
  espressi sotto (reti eterogenee, firewall, mapping etc) o sono
  concetti che andrebbero sviluppati e fatti aderire, e io ora non
  riesco davvero a vedere come la supply chain possa essere pertinente
  a un servizio di covoiturage.}

\comment{Leading from the above would be the introduction of RFID
  technologies that could be used to collect information about
  transported goods.  The usability in the Enterprise Supply Chain is
  an immediate proof of usage of our approach. From a networking point
  of view, the scientific hard point to solve are the mapping
  overlay-underlay network (to avoid critical latency issues due to
  the fact that one "logical" hop may via "many" physical hops); -
  crossing firewalls; - merging different overlay networks (\eg\,
  networks of different companies); - distributing semantic queries
  toward a P2P network; - trading off between routing complexity and
  of routing method for exhaustive discovery;- dealing with
  heterogeneous underlay networks (IP, Wifi, Zigbee, GPRS), just to
  mention a few.  Another potential application would be in the
  provisioning of supermarket companies.}


 
\bibliographystyle{alpha}
\bibliography{biblio}

\end{document}

