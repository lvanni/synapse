\documentclass{llncs}
\usepackage{times,url}
\usepackage{amsmath,amssymb,amsfonts,bbm}
\usepackage{graphicx,epsfig,epsf}
\usepackage{alltt}
\usepackage{url}
\usepackage{cite}
%\usepackage{babel}
\usepackage[utf8]{inputenc}
\usepackage[T1]{fontenc}
\usepackage{multirow}
 
   
% mix and mac 
%\usepackage[latin1]{inputenc}
\usepackage{algorithm}
\usepackage[usenames]{color}

 

%%%
\newcommand{\ie}{i.e.}
\newcommand{\eg}{e.g.}
\newcommand{\vs}{vs.}
\newcommand{\key}[1]{{\small {\sf #1}}}



\newcommand{\verbi}[1]{{\small\texttt{#1}}}
\newcommand{\comment}[1]{}


%% Intra nossss
\newcommand{\Ce} [1]{{[\color{cyan}{Cdric}}: {#1}]}
\newcommand{\Gi} [1]{{[\color{green}{Gigi}}: {#1}]}
\newcommand{\La} [1]{{[\color{cyan}{Laurent}}: {#1}]}
\newcommand{\Vi} [1]{{[\color{blue}{Vincenzo}}: {#1}]}

\newcommand{\RESP} [1]{{[\color{red}{RESP: #1}]}}




%% High penalties for line and paragraph-breaking [Dan]
\pretolerance=2000 \binoppenalty=2000 \relpenalty=1500
%\interlinepenalty=150 \predisplaypenalty=10000 \postdisplaypenalty=400
\hbadness=5000 \hfuzz=2pt 



%%%% FORMATTING
\newcommand{\rew}[1]   {\hspace{-#1mm}}
\newcommand{\fwd}[1]   {\hspace{#1mm}}
\newcommand{\down}[1]  {\vspace{#1mm}}
\newcommand{\up}[1]    {\vspace{-#1mm}} %je l'ai du decommenter parce que il cassait la compil


\usepackage{algorithm}

\newfloat{algo}{thp}{}
\floatname{algo}{Algorithm}

\newcommand{\R}{\mathtt{root}}

\newcommand{\TRUE}{\mathtt{true}}
\newcommand{\FALSE}{\mathtt{false}}
%\newcommand{\R}{\mathtt{R}}
%\newcommand{\C}{\mathcal{C}}
%\newcommand{\OR}{\mathtt{O}}
\newcommand{\IF}[1]{\textbf{if} {#1}}
\newcommand{\THEN}{\textbf{then} }
\newcommand{\ELSE}{\textbf{else} }
\newcommand{\ELSEIF}[1]{\textbf{elseif} {#1}}
\newcommand{\ENDIF}{\textbf{endif} }
\newcommand{\FOR}[1]{\textbf{for} {#1} \textbf{do }}
%\newcommand{\FOR}[2]{\textbf{for } {#1} \textbf{to } {#2} \textbf{do }}
\newcommand{\FORALL}[1]{\textbf{for all} {#1} \textbf{do }}
\newcommand{\FOREACH}[1]{\textbf{for each} {#1} \textbf{do }}
\newcommand{\WHILE}[1]{\textbf{while} {#1} \textbf{do }}
\newcommand{\REPEAT}{\textbf{repeat} }
\newcommand{\FOREVER}{\textbf{forever} }
\newcommand{\DONE}{ \textbf{done}}

\newcommand{\ALGOHEADER}[3]
{\begin{tabular}[t]{@{\extracolsep{0pt}}p{#3}l}
    \textbf{#1} &
    \begin{tabular}[t]{@{\hspace{15pt}}l}
        #2
    \end{tabular}
 \end{tabular}\smallskip
}

\makeatletter
\newcommand{\alglabel}[1]{%
  \@bsphack%
  \protected@write\@auxout{}%
         {\string\newlabel{#1}{{\the\ALGONum\LineSep \formatLine}{\thepage}}}
  \@esphack%
}
\makeatother

\newcommand{\CST}[1]      {\ALGOHEADER{Constants: }{#1}{.5in}}
\newcommand{\VAR}[1]        {\ALGOHEADER{Variables: }{#1}{.5in}}
\newcommand{\LOCALVAR}[1]   {\ALGOHEADER{Local  Variables: }{#1}{1.2in}}

\newcommand{\MACRO}[1]      {\ALGOHEADER{Macros: }{#1}{.5in}}

%\newcommand{\FUNCTION}[2]   {\textbf{function}  ${#1}$: \textbf{{#2}}}

\newcommand{\FUNCTION}[1]{\textbf{Function} {#1}}

\newcommand{\RETURN}[1]     {\textbf{return} {#1}}
\newcommand{\PROC}[1]       {\textbf{procedure} ${#1}$}
\newcommand{\INIT}[1]       {\textbf{initially} ${#1}$}
\newcommand{\ACTION}        {\textbf{actions:}}

\newcommand{\MAC}[2]
{
   ${#1} \equiv $
   \begin{tabular}[t]{@{\extracolsep{0pt}}l}
       #2
   \end{tabular}
}

\newcommand{\RCV}[1]{\textbf{upon} receipt \textbf{of} $<$#1$>$ \textbf{do}}
\newcommand{\RCVFROM}[2]{\textbf{upon} receipt \textbf{of} $<$#1$>$ \textbf{from} #2 \textbf{do}}
\newcommand{\RCVFROMSYNC}[2]{\textbf{receive} $<$#1$>$ \textbf{from} #2} %\textbf{to} #2}

\newcommand{\SEND}[2]{\textbf{send} $<$#1$>$ \textbf{to} #2}
\newcommand{\SENDSYNC}[2]{\textbf{send-sync} $<$#1$>$ \textbf{to} #2}
\newcommand{\SENDTOHOST}[1]{\textbf{send\_to\_host}$<$#1$>$}
\newcommand{\RCVFROMHOST}[1]{\textbf{receive\_from\_host}$<$#1$>$}

\newcommand{\PROCINIT}{\textbf{upon} INITIALIZATION}

\newcommand{\BEGLIST}{\begin{list}{}{\partopsep -3pt \parsep -2pt \listparindent -0pt \labelwidth .5in}}
\newcommand{\ENDLIST}{\end{list}}
\newcount\ALGOLine
\ALGOLine=-1
\newcount\ALGOLineStart
\ALGOLine=0
\newcount\ALGONum
\ALGONum=1

\newcommand{\LineSep}{.}
\newcommand{\LINESTYLE}{\scriptsize}
\newcommand{\INITALGO}[1]{\global\ALGONum=#1}
\newcommand{\INITLINE}[1]{\global\ALGOLineStart=#1}
\newcommand{\RESETLINE}{\global\ALGOLine=\ALGOLineStart}
\newcommand{\formatLine}{\ifnum\the\ALGOLine<10 0\fi\the\ALGOLine}
\newcommand{\NA}{\global\advance\ALGONum  by 1 \RESETLINE}
\newcommand{\AL}{\global\advance\ALGOLine by 1 \LINESTYLE{$\the\ALGONum$\LineSep$\formatLine$}}
\newcommand{\VL}{\ \vline\>}

\newcommand{\logreq}{{\tt logReq}}
\newcommand{\hostreq}{{\tt hostReq}}
\newcommand{\updatechild}{{\tt updateChild}}
\newcommand{\addchild}{{\tt addChild}}
\newcommand{\scanreq}{{\tt scanReq}}
\newcommand{\replicationreq}{{\tt replicationReq}}
\newcommand{\addparent}{{\tt addParent}}

\newcommand{\commonprefix}{{\bf COMMONPREFIX}}
\newcommand{\sizeof}{{\bf SIZEOF}}
\newcommand{\getpeer}{{\bf GETPEER}}
\newcommand{\getnbreplicas}{{\bf GETNBREPLICAS}}
\newcommand{\getbestreplica}{{\bf GETBESTREPLICA}}

\newcommand{\PREF}[1]{\mbox{{\sc Prefixes}}({#1})}

%%%%% ASYNC REPAIR
\newcommand{\destroy}{{\footnotesize{\bf DESTROY}}}
\newcommand{\prefix}{{\footnotesize{\bf PREFIX}}}
\newcommand{\isprefix}{\sc IsPrefix}
\newcommand{\len}{{\footnotesize{\bf LEN}}}
\newcommand{\gcp}{{\footnotesize{\bf GCP}}}
\newcommand{\inser}{{\footnotesize{\bf INSERT}}}
\newcommand{\newnode}{\sc NewNode}

\newcommand{\checkmerge}{\sc CheckMerge}
\newcommand{\checkdown}{\sc CheckDown}
\newcommand{\checkup}{\sc CheckUp}
\newcommand{\checkdef}{\sc CheckDefault}

\newcommand{\msgmerge}{\sc MsgMerge}
\newcommand{\msgdown}{\sc MsgDown}
\newcommand{\msgup}{\sc MsgUp}
\newcommand{\msgdef}{\sc MsgDefault}

\newcommand{\Bt}{B\mbox{-}tree}

%\newtheorem{theorem}{Theorem}
%% \newtheorem{hypothese}{Hypothèse}
%% \newtheorem{lemma}{Lemma}
%% \newtheorem{corollary}{Corollary}
%% \newtheorem{proposition}{Proposition}
%% \newtheorem{definition}{Definition}
%% \newtheorem{assumption}{Assumption}
%\newtheorem{remark}{Remark}

\floatname{algorithm}{Algorithm}

\newcommand{\phyreq}{{\tt phyReq}}
\newcommand{\phyreqinitiator}{{\tt phyReqInitiator}}
\newcommand{\updatesuccessor}{{\tt updateSuccessor}}
\newcommand{\host}{{\bf INSERT}}

\floatname{algorithm}{Algorithm}

\makeatletter
\providecommand*{\toclevel@algorithm}{0}
\makeatother
 
\title{CarPal: interconnecting overlay networks for a community-driven
  shared mobility\thanks{Supported by AEOLUS FP6-IST-15964-FET
    Proactive.}}
%
\author{Vincenzo Ciancaglini \and Luigi Liquori \and Laurent Vanni}

\institute{INRIA Sophia Antipolis M\'editerran\'ee, France\\
  Email: {\tt firstName.lastName@sophia.inria.fr} }

\pagestyle{plain}
\sloppy

\begin{document}

  
\maketitle
  
\begin{abstract} 
  Car sharing and car pooling have proven to be an effective solution
  to reduce the amount of running vehicles by increasing the number of
  passengers per car amongst medium/big communities like schools or
  enterprises.  However, the success of such practice relies on the
  community ability to effectively share and retrieve
  information about travelers and itineraries.  Structured overlay
  networks such as Chord have emerged recently as a flexible solution
  to handle large amount of data without the use of high-end servers,
  in a decentralized manner.  In this paper we present CarPal, a
  proof-of-concept for a mobility sharing application that leverages a
  Distributed Hash Table to allow a community of people to
  spontaneously share trip information without the costs of a
  centralized structure.  The peer-to-peer architecture allows
  moreover the deployment on portable devices and opens new scenarios
  where trips and sharing requests can be updated in real time.  Using
  an original protocol already developed that allows to
  interconnect different overlays/communities, the success rate
  (number of shared rides) can be boosted up thus increasing the
  effectiveness of our solution. Simulations results are shown to give
  a possible estimate of such effectiveness.
 
  \smallskip {\bf Keywords.} Peer to peer, overlay networks, case
  study, information retrieval, car sharing.
\end{abstract}





\section{Introduction\label{sec:introduction}}

\subsection{Context}
\Gi{Responsable}
\Vi{Co-responsable}
\subsection{Problem overview}
\Gi{Responsable}
\Vi{Co-responsable}
\subsection{Contributions}
\Gi{Responsable}
\Vi{Co-responsable}
\subsection{Outline}
\Gi{Responsable}
\Vi{Co-responsable}

\section{Interconnecting multiple networks\label{sec:link}}

\Gi{Responsable}
\Vi{Co-responsable}
\subsection{Problem with non interconnection}
\Gi{Responsable}
\Vi{Co-responsable}
\subsection{Synapses State of the Art}
\Gi{Responsable}
\Vi{Co-responsable}
\subsection{New architecture}
\Vi{Responsable}
\La{Co-responsable}


\section{Application architecture \label{sec:architecture}}

\subsection{Application principles}
%
One of the most important features for a car share application is to
be able to maximize the chances of finding a match between one driver
and one or more travelers. From this comes the choice of arranging
the database by communities in order to put in touch people who most
likely share the same traveling patterns in space and time (\eg\ work
for the same company, attend the same university and so on). Another
important aspect is to be able to update the planned itinerary
information as quick as possible so that a last minute change in plans
can be easily managed and updated and may eventually lead in finding a
new match.
 
For the above reasons, CarPal has been intended as a desktop and
mobile application running on a peer-to-peer overlay network. This
allows a community of people to spontaneously create its own travel DB
(which, as it will be shown later, can be interconnected with siblings
communities) and manage it in a distributed manner.  Moreover, it
constitutes a flexible infrastructure where,by being deployed on
connected mobile devices, it will be possible to develop more advanced
info-mobility solutions that might take into account the position of
the user/vehicle (via the internal GPS), geographically-aware network
discovery or easy network join or vehicle tracking through checkpoints
with the use of Near Field Communications technologies~\cite{NFC}.

\subsection{CarPal in a nutshell}
%
The working principle is simple: a user running a CarPal client on his
mobile or desktop will connect to one or more communities to which he
is member (\ie\ he has been invited or his request has been accepted).
Tow operations are then available, namely $(i)$ publishing a new
itinerary and $(ii)$ finding a matching itinerary.

\noindent{\bf Publishing a new itinerary.}
%
When a CarPal user has a one time or recurring trip that he wants to
optimize cost-wise he can publish his route in the community in hope
to find someone looking for a place in the same route and time window
to share the ride with. A planned itinerary is usually composed by the
following data:
%
\begin{itemize}
\item \emph{Trip date and number of repetitions} in case is a
  recurring trip;
\item \emph{Departure place and arrival place}, whose representation
  is critical since a too high granularity might lead to miss similar
  results;
\item \emph{Departure time} or at least an estimate given by the
  user;
\item \emph{Arrival time} or at least an estimate given by the user;
\item \emph{Number of available seats} to be updated when another
  passenger asks for a place;
\item \emph{Contact}, usually an e-mail or a telephone number;
\item \emph{Information}, other useful information, \ie\ animal
  allergies, women-only car etc.
\end{itemize}
%
Moreover, from a functional point of view, a trip \eg\ from a place A
to a place D may include several checkpoints, meaning that the user
offering his car can specify one or more intermediate places where he
is willing to pick up or leave a passenger.

 Once the user has inserted all the needed data (date, departure,
arrival, times, seats and optional checkpoints), the trip is 
decomposed in all the possible combinations: for example, a trip 
containing the legs A-B-C-D (where B and C are checkpoints specified 
by the user) will generate the combinations A-B and A-C and  
A-D and  B-C and  B-D and  C-D. This
operation is commonly known as \emph{Slice and Dice}.  Since the
number of possible combinations can increase exponentially with the
number of checkpoints, there is a software limitation to 3 maximum
stops in the trip. Each combination is then stored in the DHT as an
individual segment; moreover all the segments who don't start form A
are marked as estimated in departure time since, given a trip made of
different checkpoints, only the effective departure time can be
considered reliable, the others being subject to traffic conditions
and contingencies. Geographic and time information must be encoded in
such a way to be precise enough to be still relevant for our purposes
(someone leaving from the same city but 10 km far is not a useful
match) yet broad in a way that a punctual query will not skip
important results.

Until we find a feasible way to perform semantically relevant range
queries, departure/arrival hotspots can be added in a ``social''
style. A driver departing from a certain spot for the first time will
be offered a map where to mark his spot of departure, inviting the
user to use an already defined spot nearby in case. Finally,
concerning time approximation, a 20 minutes window is used to
approximate departure times. Both during an insertion or a query,
anything within the 0-19 minutes interval will be automatically set at
10 minutes, 20-39 will be set at 30 minutes and 40-59 at 50.

\noindent{\bf Finding a matching itinerary and one seat.}
%
A user wishing to find a ride can perform a search by inserting the
following information:
%
\begin{itemize}
\item \emph{Date} of the trip;
\item \emph{Departure} place and time (picked on a map between the
  proposed points;
\item \emph{Arrival} place and wished time, picked in the same
  manner as the departure.
\end{itemize}
%
To increase the chances of finding a match, only part of the search
criteria can be specified, allowing \eg\ to browse for all the trip
leading to the airport in a certain day disregarding the departure
time (giving the user the chance of finding someone leaving the hour
before) or the departure point (giving the user, in case of nobody
leaving from the same place as him, to find someone leaving nearby to
join with other means of transportation).  Moreover it is possible to
specify checkpoints in the search criteria too, in order to have the
system look for multiple segments and create aggregated responses out
of publications from multiple users.

\noindent{\bf Negotiation.}
%
Once the itinerary has been found it will be possible to contact the
driver in order to negotiate and reserve a seat. If the trip is an
aggregation of different drivers' segments all of them will be
notified through the application. The individual trip records will
then be updated by decreasing the available seats number.

 
\begin{table}[!t]
  {\small \begin{center}
    \key{
      \begin{tabular}{|c|c|c|c|}
        \hline
        \key{\bf Criteria} &\key{\bf Key} & \key{\bf Value} & \key{\bf Trip Grouped by} \\
        \hline
        1 & TR|TRIP\_ID & $\clubsuit$ & Individual \\\hline
        2 & T|DATE|DEP|TOD|ARR|TOA &list[TRIP\_ID] 
        & Dep \& Arr \& Time\\ \hline   
        3 & DA|DATE|DEP|ARR & list[TRIP\_ID] 
        & Dep \& Arr\\ \hline
        4 & D|DATE|DEP & list[TRIP\_ID] 
        & Dep\\ \hline
        5 & A|DATE|ARR & list[TRIP\_ID] 
        & Arr\\ \hline
        6 & U|USER\_ID & list[TRIP\_ID] 
        & User id\\ \hline
      \end{tabular}

     \smallskip where $\clubsuit$ = \key{[DATE,DEP,TOD,ARR,TOA,SEATS,REF,PUB]}
    }
  \end{center}}
  \caption{Different data structures stored in the DHT for each entry}
  \label{t:DHTvalues}
\end{table}

\subsection{Encoding CarPal in a DHT }
%
The segments are stored in the DHT according to Table
\ref{t:DHTvalues}.
%
Since we are not able yet to perform useful range queries on the DHT,
multiple keys are inserted or updated for each entry, representing
sets of trips grouped according to different criteria:
%
\begin{enumerate}
\item Is the actual trip record, associated to a unique
  \key{TRIP\_ID}, that will be updated, for example, when someone book
  a seat.  The information stored concerns trip date \key{DATE},
  departure place \key{DEP} and time \key{TOD}, arrival place
  \key{ARR} and time \key{TOA} number of available seats \key{SEATS}
  (or cargo space, in case of shared goods transportation), a
  reference to contact the driver \key{REF}, and if the trip has to
  be public or not \key{PUB}.  Depending on the needs more information
  can be appended to this record;
\item Represents a set of trips having the same date,
  departure/arrival point and departure/arrival time. The key is
  created by concatenating the token \key{T}, trip date \key{DATE},
  departure place \key{DEP}, departure time \key{TOD}, arrival place
  \key{ARR} and arrival time \key{TOA}.  As value is the list of
  \key{TRIP\_ID} pointing to the trip records. The key is created by
  appending the token \key{TR} to the \key{TRIP\_ID};
\item Is a set of trips grouped by date, departure and arrival place.
  It will be used to query in one request all the trips of the day on
  a certain itinerary. The key to store them in the DHT is
  consequently made by appending the token \key{DA}, trip date,
  departure and arrival point;
\item[4-5.] Are two sets of trips arranged by day and by departure
  or arrival point. The key is therefore made by concatenating either
  the token \key{D} (for departure) or \key{A} (for arrival) to the
  \key{DATE} and departure or arrival point \key{DEP} or \key{ARR}.
  This key can be used \eg to query in one request all the trips of
  the day leaving from a company or all the trips of the day heading
  to the airport; \setcounter{enumi}{5}
\item Is a set of trips for a given user. The key is therefore the
  token \key{U} prepending the \key{USER\_ID} itself.
\end{enumerate}

\subsection{Network architecture}
%
The overlay chosen for the proof of concept is Chord~\cite{chord}
although other protocol could be used to leverage the locality of the
application or a more direct geographical mapping (see Section
\ref{sec:semantic}).  Even on a simple Chord mechanisms to ensure
fault tolerance can be put in place, like data replication using
multiple hash keys or request caching.  To allow a new community to be
start up, a \emph{public tracker} has been put in place on the
Internet. The public tracker is a server whose tasks can be resumed as
following:
%
\begin{itemize}
\item It allows the setup of a new community by registering the IP of
  certain reliable peers, in a YOID-like
  fashion~\cite{francis2000yoid};
\item It acts as a central database of all the communities, keeping
  track of them and their geographical position;
\item consequently, it can propose nearby overlays to improve the
  matches by placing co-located peers;
\item It acts as a third party for the invitation of new peers into an
  overlay;
\item It can provide statistical data about the activity of n overlay,
  letting a user know if a certain community has been active lately
  (and thus it's worth joining);
\item It stores all the placeholders set as departure and arrival
  points, in order to avoid having similar routes not matching because
  of 2 different geographical markers for the same spot;
\item It act as the entry point to download the application and get
  updates.
\end{itemize}

\subsection{Enhancing the architecture}
%
It is clear that a user might take advantage of nearby communities
other than his. In case of an unsuccessful query or upon explicit
request, it is possible to access nearby networks by asking co-located
nodes in his community to reroute the query.  To do this, synapse
nodes are connected, in parallel to their actual overlays, to a common
Control Network that handles 2 different data structures (a KeyTable
and a CacheTable) used to manage inter-overlay routing. Both are
implemented as DHTs on a global overlay participated by every node of
every networks.

\noindent{\bf The Key Table}
%
is responsible for storing the unhashed keys circulating in the
underlying overlays.  When a synapse performs a \key{GET} that has to
be replicated in other networks, it makes the unhashed key available
to the other synapses through the Key Table. The key is stored using
an index formed by a networks identifier as a prefix and the hashed
key itself as a suffix. This way when a synapse on the overlay with
\eg\ \key{ID = A} will have to replicate \eg\ \key{H(KEY) = 123}, it
will be able to retrieve, if available, the unhashed key from the
KeyTable by performing a get of \key{A|123}.

\noindent{\bf The Cache Table}
%
is used to implement the replication of get requests, cache multiple
responses and control the flooding of foreign networks.  It stores
entries in the form of \key{[H(KEY),TTL,[NETID],[CACHE]]}.  In a
nutshell: \key{NETID} are optional and used to perform selective
flooding on specific networks.  When another synapse receives a
\key{GET} requests, it checks if there is an entry in the Key Table
(to retrieve the unencrypted key), and an entry in the Cache Table; if
so, it replicates the \key{GET} in the \key{[NETID]} networks he is
connected to, or in all his networks if no \key{[NETID]} are
specified. All the responses are stored in the \key{[CACHE]} and only
one is forwarded back, in order not to flood other nodes having
performed the same request. A \key{TTL} is specified to manage the
cache expiration and block the flooding of networks.  When the synapse
originating the request receives the first response, it can retrieve
from the Cache Table the rest of the results.  The cached responses
should be sent back with the associated \key{NETID}. This can allow a
with time to define a strategy of selective flooding to the networks
who are better responding to a synapse request.

\noindent{\bf Inter-overlay routing algorithm} is performed hen a peer
wish to perform query: before routing the request in his own community
it adds an entry in the KeyTable, containing the unhashed key to be
searched, and an empty entry in the CacheTable.  When an synapse in
the first overlay receives the request, it looks for the unhashed key
in the KeyTable and the corresponding entry in the CacheTable. If
those are found, the co-located synapse will query for the same key in
all his communities and store the results in the CacheTable, in order
not to pollute the network with too many results. The requesting peer
in the first network will then collect the results from the
CacheTable.

\noindent{\bf Controlling the data.}
%
Since different CarPal overlays use different hash functions to map
their keys a first level of privacy and control is guaranteed in case
a community wish to have some control over the visibility of their
information. At the present state there are 2 possible scenarios for
accessing the data:
%
\begin{itemize}
\item A user can search for trips mark as both public and private in
  every overlay he's directly connected to. As previously stated, the
  connection to an overlay happens via invitation through a mechanism
  similar to certain social networks;
\item If certain nodes of his own networks are member of other
  overlays, they can act as synapses and route queries from one
  network to another. However only the trips marked as ``public'' will
  be made available to a foreign request.
\end{itemize}


\section{A Running example \label{sec:proof}}

\La{Responsable}
\Vi{Co-responsable}
\subsection{Application description}
\subsection{Screenshots}





\section{Experimental results\label{sec:experiences}}

\subsection{Simulation settings}
\Vi{Responsable}
\La{Co-responsable}

\Vi{take it as it is from the synapse paper}
\subsection{Simulation results}
\Vi{Responsable}
\La{Co-responsable}
\Gi{Co-co-responsable}

\Vi{Take results from synapse paper}
\subsection{Results interpretation}
\Gi{Responsable}
\Vi{Co-responsable}

\Gi{Good luck on that :-D} 

\section{Conclusion and further work\label{sec:conclusion}}

Among the potential improvement, we shortly mention a few ones.
%
\subsection{Improved network bootstrap and community discovery}
%
At the present state a new community can be setup or joint by passing
through the tracker. This keeps track of community activities, their
location, handles the join negotiation and restrictions and can
suggests nearby communities that could be joined, however it
constitutes also a centralized point of failure for all the
communities. To improve further more the mechanism the following
solutions can be put in place:
%
\begin{itemize}
\item Assuming that a community/overlay could very likely reside on
  the same network infrastructure (i.e. the enterprise intranet) a
  discovery protocol can be put in place leveraging existing
  technologies like Avahi \cite{avahi:website} to discover new peers
  or new networks to join;

\item Peer caching could be used to reconnect to previously connected
  peers whose activity is known to be reliable;

\item An invitation to a new community could be handled physically via
  an NFC transaction. A user with an NFC enabled phone could be
  invited by another user by simply swiping the phones
  together. Furthermore this could be an additional guarantee of a
  user ``reliability'' as the new participant needs to be known and
  met by an existing member;

\item The community database could be as well stored in the DHT
  itself, meaning by this that new communities could simply be
  discovered through specific requests routed through existing
  synapses to other networks in a ping-like way.
\end{itemize}


\subsection{Semantic queries and specialized protocols}\label{sec:semantic}
%
It appears clear that the current approach suffers from the limitation
of a simple key-value approach. Such approach does not fit well into
an application that finds her strength in the possibility of
performing searches according to many different criteria.  The
adoption of a semantic hash function (such as \cite{SemanticHash})
would allow to cluster in nearby peers information semantically close
(\ie\ trips heading to sibling destinations or taking place in the
same time window).  Needless to say, with such hashing in place the
adoption of an overlay protocol more suited to range queries (like
P-Ring \cite{P-Ring}, P-Grid\cite{P-Grid} or Skipnet
\cite{Harvey03skipnet:a}) might lead the semantically significant
range queries, where, for example, departure and arrivals can be
geographically mapped and queried with a certain range in Km.

Another possible improvement (currently under study) would be to use a
DHT protocol more suited for geo-located information. CAN \cite{CAN})
in a 2D configuration is a first example of how this could be
achieved. Mapping CAN's Cartesian space over a limited geographic area
(like in Placelab \cite{PlaceLab}) could ease the query routing and
eventually provide some strategic points to place synapsing nodes.

\subsection{Overlay-underlay mapping optimizations}
%
From a networking point of view, the scientific hard point to solve is
the overlay-underlay network mapping (to avoid critical latency issues
due to the fact that one ``logical'' hop may via ``many'' physical
hops); The issue is being investigated and could involve for example
the use of several always on-servers to be used to ``triangulate'' the
position of a peer over the internet (according to latency metrics)
and cluster together ``nearby'' peers (whereas nearby will mean sharing
similar latencies to the same given servers).  Another issue would be
to make the service firewall-resilent by implementing TCP Punch-hole
techniques in the peer engine and in the tracker.


\subsection{Backward compatibility with other Carpool services}
%
At the present time we have developed a CarPal service that is
suitable to interconnecting Carpool services whose standards are open,
collaborative and based on our software. Unfortunately this is not
always the case. This makes a backward compatibility problem in case
we want to connect existing services. To take into account this case,
the Synapse protocol we have developed allows also a, so called,
\emph{black box} variant, that is suitable to interconnecting overlays
that, for different reasons, are not collaborative at all, in the
sense that they only route packets according to their proprietary and
immutable protocol.  Being the black box more of a meta-protocol
running on top of existing, and not necessarily peer-to-peer,
protocols, we can imagine strategically placed Synapse nodes being
responsible of querying existing web services and returning the
corresponding information as if they were coming from a foreign
network. This open new scenarios were multimodality is easily
integrated and made available for the nearby communities. The system
needs to be properly designed in order to avoid a situation where too
many peers act as a Distributed Denial Of Service, but the current
infrastructure make it easily feasible.

\subsection{User rating, social feedback}
%
Being CarPal an application based on user-generated content and
designed to put in touch people not necessarily acquainted to each
other, it is important to implement some social feedback and security
mechanism to promote proactive and good behaviors by the users.
Borrowing from today's most successful web applications, 2 solutions
can be imagined:
%
\begin{itemize}
\item A user rating should be put in place in order, for example, to
  allow passengers to evaluate a driver's punctuality, behavior and
  driving skills, and vice-versa. This feedback, similar to what
  systems like Ebay already have since several years, can help
  maintaining a high level of quality in the service by giving an
  immediate picture of a driver or passenger's reliability;

\item Some points can be assigned to users based on their activity in
  the community. The more a user will be proactive by publishing or
  subscribing to new trips in an overlay, the more ``karma points'' he
  will receive.  A similar approach can be verified in Social News
  website like Digg~\cite{Digg} or Reddit~\cite{Reddit} and has become
  pretty common in today's social media.  With the deployment and
  integration of new distributed services, these points could act as a
  ``virtual cash'' and grant access to features normally reserved to
  paying customers, thus motivating drivers and passengers to keep a
  community alive.
\end{itemize}

\subsection{Other potential applications}
%
The Car sharing/pooling is not the exclusive applicative field the
overlay network technology we designed; with the same final objective
of minimizing traffic, pollution and energy a service interconnecting
transportation companies Information Systems could be envisaged. A
``BoxPal'' system could be easily build using the same overlay network
technology: the only difference being the (more difficult) 3D
bin-packing combinatorial algorithms employed instead of simple
matching of drivers/cars/itinerary/car places.

%%%%%
%%%%% SPARE MATERIAL  C O M M E N T !!!
%%%%%
\comment{ TUTTO RFID MATERIAL IN VRAC DA POMPARE....VEDI ANCHE NEL PDF
CHE TI HO MANDATO
%
Another potential application would be the introduction of the RFID
technologies that could be used to collect information about locations
and possessions of people (and hence about their preferences, possible
medical conditions, ...) without them knowing about it generated fears
in the general public.
%
\Gi{DS = discovery services}
%
The usability in the Enterprise Supply Chain is an immediate proof of
usage of our approach. From a networking point of view, the scientific
hard point to solve are the mapping overlay-underlay network (to avoid
critical latency issues due to the fact that one ``logical'' hop may
via ``many'' physical hops); - crossing firewalls; - merging different
overlay networks (\eg\ , networks of different companies); -
distributing semantic queries toward a P2P network; - trading off
between routing complexity and of routing method for exhaustive
discovery;- dealing with heterogeneous underlay networks (IP, Wifi,
Zigbee, GPRS), just to mention a few.
%
Our overlay technology has various originalities that render it very
interesting from a social viewpoint. First, its particular knowledge
sharing approach will permit databases and DSs to support much more
cooperation between people than currently. Second, it will also permit
people to design their own access control or usage rules and associate
them to the information they own. Third, thanks to our DSs, people (or
their personal agents) will better be able to find which databases
have stored information about them (then, if these persons are not
happy with that storage, they can legally ask a database
server/manager to remove or update such information). Fourth,
conversely, if these persons have their own personal databases where
they have described access control and usage rule for some kinds of
information about themselves (\eg, their age or their weight), thanks
to our DS, database servers could find such rules and respect
them. This last point is certainly a naive ideal but could be checked
by people thanks to the third point and then made well known by them
or rewarded in some other way.
%
Systems such as ``Google Latitude'' allows people to let persons they
trust track them. A flexible and secure DS could be uhypothesis.
%
 One objective of RFID3S is to build a P2P overlay network where peers
contain fragments of RFID databases. Due to the heterogeneity of the
network and due to the presence of peers playing different roles and
of SW and HW barriers between companies (very often competitors) we
plan to use a - so called - hybrid overlay networks. A popular
topology has a super-peer serving as a server for a subset of the
peers, as in well know Skype VOIP application.
%
Towards a discovery service in a scalable, secure and flexible
Internet of things including RFID and NFC devices.  The final
objective of RFID3S is to build a hybrid overlay network where
security, scalability and semantic expressivity of queries cohabit
without tears for companies and end-users. Using a P2P infrastructure
for modeling an ONS of RFID/NFC in presence of different kinds of
peers and different kind of underlay network is probably one of the
more interesting challenges of the project.  Routing also complex
queries towards this network in a secure way is also worth of
interest.
%
The usability in the Enterprise Supply Chain is an immediate proof of
usage of our approach. From a networking point of view, the scientific
hard point to solve are the mapping overlay-underlay network (to avoid
critical latency issues due to the fact that one ``logical'' hop may via
``many'' physical hops); - crossing firewalls; - merging different
overlay networks (e.g., networks of different companies); -
distributing semantic queries toward a P2P network; - trading off
between routing complexity and of routing method for exhaustive
discovery;- dealing with heterogeneous underlay networks (IP, Wifi,
Zigbee, GPRS), just to mention a few.  }



\comment {spostato tutto in fondo. Cominciamo dal tecnico e andiamo sul funzionale.}

\comment{Questo te l' ho commentato perche' o i concetti sono gia'
  espressi sotto (reti eterogenee, firewall, mapping etc) o sono
  concetti che andrebbero sviluppati e fatti aderire, e io ora non
  riesco davvero a vedere come la supply chain possa essere pertinente
  a un servizio di covoiturage.}

\comment{Leading from the above would be the introduction of RFID
  technologies that could be used to collect information about
  transported goods.  The usability in the Enterprise Supply Chain is
  an immediate proof of usage of our approach. From a networking point
  of view, the scientific hard point to solve are the mapping
  overlay-underlay network (to avoid critical latency issues due to
  the fact that one "logical" hop may via "many" physical hops); -
  crossing firewalls; - merging different overlay networks (\eg\,
  networks of different companies); - distributing semantic queries
  toward a P2P network; - trading off between routing complexity and
  of routing method for exhaustive discovery;- dealing with
  heterogeneous underlay networks (IP, Wifi, Zigbee, GPRS), just to
  mention a few.  Another potential application would be in the
  provisioning of supermarket companies.}


 
\bibliographystyle{alpha}
\bibliography{biblio}

\end{document}

