% intro.tex

%\subsection{Context}
%
\noindent {\bf Context.}
%
The inter-connection of overlay networks has been recently identified
as a promising model to cope with today's Internet issues such as
scalability, resource discovery, failure recovery or routing
efficiency, in particular in the context of information
retrieval. Some recent researches have focused on the design of
mechanisms for building bridges between heterogeneous overlay networks
for the purpose of improving cooperation between networks that have
different routing mechanisms, logical topologies and maintenance
policies. However, more comprehensive approaches of such
inter-connections for information retrieval and both quantitative and
experimental studies of its key metrics, such as satisfaction rate or
routing length, are still missing. During the last decade, different
overlay networks were specifically designed to answer well-defined
needs such as content distribution through unstructured overlay
networks such as Kazaa or through structured networks, mainly under
the shape of Distributed Hash Tables~\cite{CAN,Chord,Pastry},
publish/subscribe systems~\cite{castro_scribe:large-scale_2002,
  LBCC08}.
% , networked virtual environment~\cite{knutsson_peer-to-peer_2004}.

% , transactional key/value store~\cite{schtt_scalaris:_2008}, or
% video
% streaming~\cite{xinyan_zhang_coolstreaming/donet:data-driven_2005}.
% Overlay networks' topologies span over structured, unstructured or
% even hybrid ones: they have completely different properties in terms
% of routing efficiency, exhaustivity, scalability, and reliability
% (refer to~\cite{P2PBOOK} for a survey).

%\subsection{Problem}
%
\noindent {\bf An overview of the current problem.} 
Many disparate overlay networks may not only simultaneously co-exist
in the Internet but also compete for the same resources on shared
nodes and underlying network links. One of the problems of the overlay
networking area is how heterogeneous overlay networks may
\emph{interact} and \emph{cooperate} with each other. Overlay networks
are heterogeneous and basically unable to coo\-perate each other in an
effortless way, without merging, an operation which is very costly
since it not scalable and not suitable in many cases for security
reasons.  However, in many situations, distinct overlay networks could
take advantage of cooperating for many purposes: collective
performance enhancement, larger shared information, better resistance
to loss of connectivity (network partitions), improved routing
performance
in\,terms\,of\,delay,\,throughput\,and\,packets\,loss,\,by,\,for
instance,\,cooperative\,forwarding\,of flows.

As a basic example, let us consider two distant databases. One node of
the first database stores one $(key, value)$ pair which is searched by
a node of the second one. Without network cooperation those two nodes
will never communicate together.  As another example, we have an
overlay network where a number of nodes got isolated by an overlay
network failure, leading to a partition: if some or all of those nodes
can be reached via an alternative overlay network, than the partition
``could'' be recovered via an alternative routing.

In the context of large scale information retrieval, several overlays
may want to offer an aggregation of their information/data to their
potential common users without losing control of it. Imagine two
companies wishing to share or aggregate information contained in their
distributed databases, obviously while keeping their proprietary
routing and their exclusive right to update it.  Finally, in terms of
fault-tolerance, cooperation can increase the availability of the
system, if one overlay becomes unavailable\,the
global\,network\,will\,only\,undergo\,partial\,failure\,as\,other\,distinct\,resources\,will\,be\,usable.

We consider the tradeoff of having one \vs\ many overlays as a
conflict without a cause: having a single global overlay has many
obvious advantages and is the \textit{de facto} most natural solution,
but it appears unrealistic in the actual setting. In some optimistic
case, different overlays are suitable for collaboration by opening
their proprietary protocols in order to build an open standard; in
many other pessimistic cases, this opening is simply unrealistic for
many different reasons (backward compatibility, security, commercial,
practical, etc.). As such, studying protocols to interconnect
collaborative (or competitive) overlay networks is an interesting
research vein.% that is worth pursuing.

% Recent research has focused on the design of network
% inter-connection mechanisms building bridges between heterogeneous
% overlay networks for cooperation, but simple protocols describing
% this cooperation, and comprehensive quantitative and empirical
% studies of metrics such as satisfaction rate or routing length in
% the context of scalable information retrieval are still missing.

% \subsection{Applications}
% ... social networks in p2p, interconnecting distributed data bases
% in R or RW from other databases, anti censorship apps, routing over
% sw/hw barriers, brain simulations....

%\subsection{Contribution}
%
\noindent {\bf Contribution. } The main contribution of this paper is
to introduce, simulate and experiment with \emph{Synapse}, a scalable
protocol for information retrieval over the inter-connection of
heterogeneous overlay networks. The protocol is based on co-located
nodes, also called \emph{synapses}, serving as low-cost natural
candidates for inter-overlay bridges.  In the simplest case (where
overlays to be interconnected are ready to adapt their protocols to
the requirements of interconnection), every message received by a
co-located node can be forwarded to other overlays the node belongs
to. In other words, upon receipt of a search query, in addition to its
forwarding to the next hop in the current overlay (according to their
routing policy), the node can possibly start a new search, according
to some given strategy, in some or all other overlay networks it
belongs to. This obviously implies to providing a Time-To-Live value
and detection of already processed queries, to avoid infinite loop in
the network, as in unstructured peer-to-peer systems.

% % social issues dropped for the moment ...
% social issues We also study interconnection policies as the explicit
% possibility to rely on \emph{social} based strategies to build these
% bridges between distinct overlays; nodes can invite or can be
% invited.

In case of concurrent overlay networks, inter-overlay routing becomes
harder, as intra-overlays are provided as some black boxes: a
\emph{control} overlay-network made of co-located nodes maps one
hashed key from one overlay into the original key that, in turn, will
be hashed and routed in other overlays in which the co-located node
belongs to. This extra structure is unavoidable to route queries along
closed overlays and to prevent routing loops.
 

% % Hexaustivity vs non hexaustivity : TTL
% This simple move implies, for example, that if a routing inside a
% structured overlay network is exhaustive, the corresponding routing
% toward a set of structured overlay networks via the Synapse protocol
% is not; therefore a Time-To-Live (TTL) mechanism recording the
% number of physical hops must be introduced to avoid the undelivered
% packets keeps circulating forever in the inter-network.

% % Game over
% Another optimization strategy (the ``game over'' strategy) in
% Synapse allows to cut replicated routing in case a lookup will pass
% twice on the same node.

% % Social issues
% Another novelty of Synapse is the inclusion of built-in primitives
% to deal with social networking in order to give an incentive for
% cooperation between nodes. Every node can set a personal strategy to
% maximise successful routing for him and for all the citizens of its
% network. To this concern two operations are available: a node can
% join another network or can invite another node to join the current
% network, according to a peculiar strategy that gives a ``score'' to
% nodes and to networks performances.

% Consequences
Our experiments and simulations show that a small number of
well-connected sy\-napses is sufficient in order to achieve almost
exhaustive searches in a ``synapsed'' network of structured overlay
networks. We believe that Synapse can give an answer to circumventing
network partitions; the key points being that:
%
%\begin{itemize}
%\item 
$(i)$ several logical links for one node leads to as many alternative
physical routes through these overlay, and
%\item 
$(ii)$ a synapse can retrieve keys from overlays that it doesn't even know
simply by forwarding their query to another synapse that, in turn, is
better connected.
%\end{itemize}
%
Those features are achieved in Synapse at the cost of some additional
data structures and in an orthogonal way to ordinary techniques of
caching and replication.  Moreover, being a synapse can allow for the
retrieval of extra information from many other overlays even if we are
not connected with.  We summarize our contributions with the
following:
%
%\begin{itemize}
%
%\item 
$(i)$ the introduction of \emph{Synapse}, a generic protocol, which is
suitable for inter-connecting heterogeneous overlay networks without
relying on merging in presence of open/collaborative or
closed/competitive networks;
%
%\item 
$(ii)$ extensive simulations in the case of the interconnection of
structured overlay networks to capture the real behavior of such
platforms in the context of information retrieval, identify their main
advantages and drawbacks;
%
%\item 
$(iii)$ the deployment of a lightweight prototype of Synapse, called
\texttt{JSynapse} on the Grid'5000 platform\footnote{{\sf
    http://www.grid5000.fr} and {\sf
    http://open-chord.sourceforge.net}.} along with some real
deployments showing the viability of such an approach while validating
the software itself;
%
%\item 
$(iv)$ finally, on the basis of the previous item, the description and
the deployement on the Grid'5000 platform of a open source prototype,
called \texttt{open-Synapse}, based on the \texttt{open-Chord}$^4$
implementation of Chord
% \footnote{{\sf http://open-chord.sourceforge.net.}},
  inter-connecting an arbitrary number of Chord networks.
%\end{itemize}
%
The final goal is to grasp the complete potential that co-located
nodes have to offer, and to deepen the study of overlay networks'
inter-connection using these types of nodes.

% \subsection{Outline}
%
\noindent {\bf Outline.} The remainder of the paper is organized as
follows: In Section~\ref{sec:protocol}, we introduce our Synapse
protocol, declined for open/collaborative overlays (\emph{white box})
viz. closed/com\-petitive (\emph{black box}) overlays. We provide
examples.
% 
% and pseudocode. 
In Section~\ref{sec:simulations}, we present the results of our
simulations of the Synapse protocol to capture the behavior of key
metrics traditionally used to measure the efficiency of information
retrieval. In Section~\ref{sec:deployement}, we describe a deployment
of a client prototype\footnote{Code and web appendix are available
  at~{\sf http://www-sop.inria.fr/teams/lognet/synapse}.}  over the
Grid'5000 platform. Section~\ref{sec:relatedwork}, we summarize the
mechanisms proposed in the literature related to the overlay networks'
cooperation.  Conclusions and further work conclude.  Due to lack of
space, the protocol pseudocode is presented in a separate web
appendix$^5$. 



