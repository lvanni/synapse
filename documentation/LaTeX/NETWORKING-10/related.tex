% related.tex

%% Identifying future trends Independantly, authors

% In \cite{siekkinen_beyondfuture_2007}
% and~\cite{fonte_interdomain_2008}, authors have recently presented
% some architectural issues with the Internet and identified some
% future trends, one of which is the inter-communication of
% \emph{realms}\footnote{A realm is a term authors use in order to
%   identify a network instance.}. They emphasise on the
% inter-communication of logical networks, and discuss various ways to
% do so. In particular, they point out that co-existing realms should
% share nodes in some cases and should not in some others. For
% instance, networks may deliberately behave selfishly only aiming at
% the improvement of their application-level QoS by taking advantage
% of dedicated paths to other networks, \ie\ using \textit{gateways},
% acting as dedicated bridges between networks. But in many cases they
% argue that it is more natural to take advantage of nodes that are
% already part of multiple overlay networks, \ie\ using
% \emph{co-located nodes}.

%\subsection{Cooperation through hierarchy}
%% Hierarchical

\noindent{\bf Cooperation through hierarchy.} 
%
While pointing out the limits of a unique global structured overlay network
(rigidity, maintenance cost, security, etc.), several propositions
have been made over the years to build alternate topologies based on
the coexistence of smaller local overlay networks. The first approach
has been based on hierarchical systems \cite{Biersack,XuMH03}, where some
elected super-peers being are promoted to a top-level overlay network,
leading to the requirement of costly merging mechanisms to ensure a
high level of exhaustiveness. In a more general view, merging several
co-existing structured overlay networks has been shown to be a very
costly operation \cite{Datta,Haridi}.

%% Transition/Contribution
In the context of mobile ad hoc networks, Ariwheels
\cite{BCCL08,LBCC08} has been designed to provide a large variety of
services through a multi-layer overlay network, where super-peers,
called Brokers, act as servers for a subset of peers.  Ordinary peers,
called Agents, submit queries to their Broker and receive results from
it. Ariwheels provides an efficient mapping between physical devices
in the wireless underlay network and virtual entities in the overlay
network.

%\subsection{Cooperation through gateways}
%
\noindent{\bf Cooperation through gateways.}.
%
Authors in~\cite{cheng2006tdh} present two models for two overlays
to be (de)composed, known as \textit{absorption} (a sort of merging)
and \textit{gatewaying}.  Their protocol enables a CAN-network to be
completely absorbed into another one (in the case of the absorption),
and also to provide a mechanism to create bridges between DHTs (in the
case of the gatewaying). They do not specifically take advantage of a
simple assumption that nodes can be part of multiple overlays at  the
same time, thus playing the role of natural bridges.

More recently, authors in \cite{cheng_bridging_2007} propose a novel
information retrieval protocol, based on gateways, called
\emph{DHT-gatewaying}, which is scalable and efficient across
homogeneous, heterogeneous and assorted co-existing structured overlay
networks\footnote{Ex.  Two 160-bit Chord, or two 160/256-bit Chord, or
  one 160-bit Chord and one 256-CAN.}.  They argue that there is not one
preferred structured overlay network implementation, and that peers
are members of co-existing DHTs. Their assumptions are $(i)$ only some
peers support the implementations of different DHTs and $(ii)$ some
peers are directly connected to peers that are members of other DHTs,
and are called \textit{Virtual Gateways (VG))}. When a request is sent
in one overlay, and no result was found, the requester can opt to
widen his search by forwarding the original search request to nodes
which belong to other structured overlay networks (mapping the search
to the format supported by their relative overlay). A TTL value is
added to the original search in order to avoid cycles; this value is
decremented each time a request crosses a new DHT
domain. Unfortunately the evaluation of their protocol lacks precious
details and precision. It is unclear how they evaluate their protocol.

%Because VGs can be overloaded, authors
%devised a mechanism in order to distribute the mapping by electing
%more VGs (according to a specific VG determination scheme), and they
%also introduced self-organizing \emph{gateways pointers} whose roles
%are to keep track of VGs where-abouts.

%% Synergy: co-located nodes are good candidates
%% compare hierarchy and gateway to our approach
%% and state why co-located nodes are a more
%% natural choice
%\subsection{Cooperation through co-located nodes}
%% Cooperation: more hints

\noindent{\bf Cooperation through co-located nodes.}
%
Authors in \cite{kwon_synergy:overlay_2005} present Synergy, an
overlay inter-networking architecture which improves routing
performance in terms of delay, throughput and packet loss by providing
cooperative forwarding of flows between networks. Authors suggest that
co-located nodes are good candidates for enabling inter-overlay
routing and that they reduce traffic. Our approach can also be seen as
a deeper study of their concepts.

On the way of designing inter-overlay networking based on co-located
nodes, authors in~\cite{junjiro_design_2006} present algorithms which
enable a symbiosis between different overlays networks with a
specific application in mind: file sharing. They propose mechanisms
for hybrid P2P networks cooperation and investigate the influence of
system conditions, such as the numbers of peers and the number of
meta-information a peer has to keep. They provide interesting
observations on how to join a candidate network, cooperative peers'
selection, how to find other P2P networks, when to start cooperation,
by taking into account the size of the network (for instance, a very
large network will not really benefit from a cooperation with a small
network), so on and so forth. Again, a more comprehensive
understanding of this approach is missing.

%Their simulations showed the effect of
%the popularity of a cooperative peer on the search latency evaluation,
%that is the more a node has neighbors, the better, as well as the
%effect of their caching mechanism which reduces (when appropriately
%adjusted) the load on nodes (but interestingly does not contribute to
%faster search).

%% Focus on internetwork routing policies
Authors in \cite{furtado_multiple_2007} consider multiple spaces with
some degree of intersection between spaces, \ie\ with co-located
nodes. They focus on different potential strategies to find a path
to another overlay from a given overlay, \ie\ how requests can be
efficiently routed from one given overlay to another one. They
compare various inter-space routing policies by analyzing which
trade-offs, in terms of state overhead, would give the best results,
in terms of the number of messages generated and routed, the number of
hops it takes to find a result and the state overhead (\ie\ the number
of fingers a node has to keep). They provide a comparative analytical
study of the different policies. They showe that with some dynamic
finger caching and with multiple gateways (in order to avoid
bottlenecks and single points of failures) which are tactfully laid
out, they obtain pretty good performances. Their protocol focuses on the 
interconnection of DHTs, while we extend it to any kind of overlays.

% %% Babelchord
% In our previous preliminary work \cite{LTB09}, we introduced
% BabelChord, a protocol for inter-connecting Chord overlay networks
% using co-located nodes that are part of multiple Chord
% ``floors''. These nodes connect, in an unstructured fashion, several
% Chord overlays together The simulations showed that we could achieve
% pretty high exhaustivity with a small amount of those co-located
% nodes. Our current paper, in turn, focuses on the co-located nodes
% heuristic in far more details than the aforementioned work by
% providing not only a more generic protocol which enables inter-overlay
% routing that can in principle be applied to connect arbitrary
% heterogeneous overlays, but also more simulations to show the
% behaviours of such networks as well as a real implementations and live
% experiments.



% \subsection{Most related works }
% The most related works that can be compared to the current paper are:
% \cite{cheng_bridging_2007,kwon_synergy:overlay_2005,furtado_multiple_2007,
% junjiro_design_2006,LTB09}.
% In summary the key differences/similitudes with the present paper are:

% \begin{maliste}
% % \item \cite{cheng_bridging_2007}: Authors here present a protocol
% % which can handle co-located nodes as well as direct gateways. Although they
% % focus on wireless ad hoc  network, they claim that their protocol can
% % be used in wired networks too. Their insights  regarding the placement
% % and the selection of their \textit{Virtual Gateways} are fairly
% % precise.  Unfortunately the evaluation of their protocol lacks
% % precious details and precision. It is unclear  how they evaluate their
% % protocol.

% \item \cite{kwon_synergy:overlay_2005}: The Synergy internetworking
% architecture  also makes use of co-located node in order to improve
% global performances. They try to create and maintain long-lived flows
% (\ie long-lived paths) that can be used to cross  overlay
% boundaries. They provide hints on how to choose nodes which will take
% part in those flows,  and they provide simulations and real experiments from a
% prototype client.  However they do not go into details as much as we
% do in this paper regarding the algorithms that enable such overlay
% inter-connection.

% \item \cite{furtado_multiple_2007}: The authors strongly focused their
% attention  on the state overhead of the nodes participating in several
% spaces in the same time (\ie using also co-located nodes),
% and provide fairly accurate analyses and strategies to
% minimise it. They do not provide a generic protocol in the sense that
% their protocol only focus on DHT's inter-connection.

% \item \cite{junjiro_design_2006}: Authors only focus on file sharing
% whereas our protocol can be applied to any application. Even if the
% algorithms and the various observations they present are relevant
% they fail to provide any real experiments nor an in-depth analysis of
% their algorithms.

% \item \cite{LTB09}: Our current paper focuses on the co-located
% nodes heuristic in more details than the aforementioned work by
% providing not only a more generic protocol which enables
% inter-overlay routing that can in principle be applied to connect
% arbitrary heterogeneous overlays, but also more simulations to show the
% behaviours of such networks as well as a real implementation and live
% experiments.

% \end{maliste}


%  We based our work on co-located nodes and
% try to take advantage of the fact that a node participates in multiple
% overlay networks in the same time on the Internet. However, we try to
% provide a deeper analysis of such type of inter-overlay networks
% compared to other works and we also try to not only to give strong
% simulations but also live deployments.


\begin{comment}
  In this sense, and in response to overlay detractors, we argue
  %% list some coherent responses for their comments that ON are not
  %% suited for interconnecting domains
  that works like \cite{XuMK03}, \cite{EURECOM+1205} and
  \cite{zhou_balancing_2003} show efficient methods for constructing an
  overlay network while taking into account the underlying
  topology. Therefore, we can say with confidence that we do have
  mechanisms in order to ensure that the paths the packets traverse
  are not using duplicate physical links.
\end{comment}




