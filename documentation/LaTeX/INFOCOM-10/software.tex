% software.tex
Implementation of the Synapse protocol client is based on \texttt{open-chord} implementation
\cite{Bamberg-SW}. This client currently can interconnect an arbitrary number of Chord networks. This
implementation follows notation of the \cite{LTB09}, so every new Chord network is called \textit{Floor}.
Regarding \cite{Bamberg-SW} some new classes were implemented, like: Floor or MyFloor. The rest of code
is changed only to follow new data structure, that References and Entries are specific for the particular
floor. Major changes were made in main classes NodeImpl and ChordImpl, as well in the communication part,
like specific proxy classes: SocketProxy with RequestHandler and ThreadProxy with ThreadEndpoint.

Following the idea that every node potentially is a neural synapse, the decision was made not to do full
object-oriented extension of the classes, but only to change \texttt{open-chord}  implementation
\cite{Bamberg-SW}, because the new classes which should extend old classes would have  almost the
same code as the old ones and the only new thing should be calls to changed structure. So,  in this
case we have not real full object-oriented extension, just to deal with some sort the siblings
classes.

When we talk in figures, comparing to \texttt{open-chord} implementation
\cite{Bamberg-SW}, we can say that about 1000 lines were added and 1500 lines were changed.