
% intro.tex

%% Context/small intro on overlays
\subsection{Context}
The inter-connection of overlay networks has been recently identified
as a promising model to cope with today's Internet issues such as
scalability, failure recovery or routing efficiency, in particular in
the context of information retrieval. Some recent researches have
focused on the design of mechanisms for building bridges between
heterogeneous overlay networks in the purpose of improving cooperation
between networks that have different routing mechanisms, logical
topologies and maintenance policies.

However, more comprehensive approaches of such inter-connections for
information retrieval and both quantitative and experimental studies
of its key metrics such as satisfaction rate or routing length are
still missing. During the last decade, different overlay networks were
specifically designed to answer well-defined needs such as content
distribution through unstructured overlay networks such as Kazaa or
through structured networks, mainly under the shape of Distributed
Hash Tables~\cite{CAN,Chord,Pastry,symphony},
publish/subscribe systems~\cite{castro_scribe:large-scale_2002,
  LBCC08,CCL08}, networked virtual
environment~\cite{knutsson_peer-to-peer_2004}, transactional key/value
store~\cite{schtt_scalaris:_2008}, or video
streaming~\cite{xinyan_zhang_coolstreaming/donet:data-driven_2005}.

Overlay networks' topologies span over structured, unstructured or
even hybrid ones: they have completely different properties in terms
of routing efficiency, exhaustivity, scalability, and reliability
(refer to~\cite{P2PBOOK} for a survey).

\subsection{Problematic}
%
Many disparate overlay networks may not only simultaneously co-exist
in the Internet but also compete for the same resources on shared
nodes and underlying network links. One of the problems of the overlay
networking area is how heterogeneous overlays networks may interact
and cooperate with each others. Overlay networks are heterogeneous and
basically unable to cooperate each other in an effortless way, without
merging, an operation which is very costly and not suitable in many
cases (security, performance, etc.).

However, in many contexts, distinct overlay networks could take
advantage af cooperating for many purposes: collective performance
enhancement, larger shared information, better resistance to loss of
connectivity (network partitions), improved routing performance in
terms of delay, throughput and packets loss, for instance by
cooperative forwarding of flows. In the context of large scale
information retrieval, several overlays may want to offer an
aggregation of their information/data to their potential common users
without losing the control of it (imagine two companies wishing to
share or aggregate the information contained in their database,
obviously while keeping the exclusive write to update it). Finally, in
terms of fault-tolerance, cooperation can really increase the
availability of the system, if one overlay becomes unavailable, the
global network will only undergo partial failure as other distinct
resources will be usable.

% DEJA DIT !!!
% Recent research has focused on the design of network inter-connection
% mechanisms building bridges between heterogeneous overlay networks for
% cooperation, but simple protocols describing this cooperation, and
% comprehensive quantitative and empirical studies of metrics such as
% satisfaction rate or routing length in the context of scalable
% information retrieval are still missing.

%\subsection{Applications}
%\Lu{FIX. social networks in p2p, interconnecting distributed data
%  bases in R or RW from other databases, anti censorship apps, routing
%  over sw/hw barriers, brain simulations.....}


\medskip
\subsection{Contribution}
%% Contributions/what we do
The main contribution of this paper is to introduce, simulate and
experiment \emph{Synapse}, a scalable protocol for information
retrieval over the inter-connection of heterogeneous overlay
networks. Synapse is based on co-located nodes, also called
\emph{synapses}, serving as low-cost natural candidates for
inter-overlay bridges.

% The idea in a nutshell
In a nutshell, every message received by a co-located node can be
forwarded to other overlays the node belongs to. In other words, upon
receipt of a search query, in addition to its forwarding to the next
hop in the current overlay (according to the routing policy of the
overlay, the node can possibly start a new search, according to some
given social-based strategy, in some or all other overlay networks it
belongs to. This obviously implies to provide a Time-To-Live value and
detection of already processed queries to avoid infinite loop in the
network, as in unstructured peer-to-peer systems. Another novelty of
our design is the explicit possibility to rely on \emph{social}
primitives to build these bridges between distinct overlays. A node
can \emph{invite} another node to join its own overlay, based on some
particular strategy.

% Hexaustivity vs non hexaustivity : TTL
% This simple move implies, for example, that if a routing inside a
% structured overlay network is exhaustive, the corresponding routing
% toward a set of structured overlay networks via the Synapse protocol
% is not; therefore a Time-To-Live (TTL) mechanism recording the number
% of physical hops must be introduced to avoid the undelivered packets
% keeps circulating forever in the inter-network.
% % Game over
% Another optimization strategy (the ``game over'' strategy) in Synapse
% allows to cut replicated routing in case a lookup will pass twice on
% the same node.

% % Social issues
% Another novelty of Synapse is the inclusion of built-in primitives to
% deal with social networking in order to give an incentive for
% cooperation between nodes. Every node can set a personal strategy to
% maximise successful routing for him and for all the citizens of its
% network. To this concern two operations are available: a node can join
% another network or can invite another node to join the current
% network, according to a peculiar strategy that gives a ``score'' to
% nodes and to networks performances.

% Consequences

% Our experiments and simulations show that a small number of
% well-connected synapses is enough to achieve almost exhaustive
% searches in a ``synapsed'' network of structured overlay networks. We
% believe that Synapse can give an answer to circumvent network
% partitions; the rationale is simple: as many logical links a node can
% have, as many alternative physical routes through these overlay we can
% best dispose. This feature is achieved in Synapse at the cost of some
% additional data structures and in an ortogonal way to ordinary
% techniques of caching and replication.

Our contributions are the following:
%
\begin{itemize}
%
\item the introduction of \emph{Synapse}, a generic protocol, which is
  suitable for inter-connecting heterogeneous overlay networks without
  relying on merging;
%
\item extensive simulations in the case of the interconnection of
  structured overlay networks to capture the real behavior of such
  platforms in the context of information retrieval, identify their
  main advantages and drawbacks;

%
\item the deployment of a lightweight prototype of Synapse, called
  \texttt{JSynapse} on the Grid'5000 platform\footnote{{\sf
      http://www.grid5000.fr.}} along with some real deployments
  showing the viability of such an approach while validating the
  software itself;
%
\item finally, on the basis of the previous item, the description of a
  real open source prototype, called \texttt{open-Synapse}, based on
  the \texttt{open-Chord} implementation of Chord\footnote{{\sf
      http://open-chord.sourceforge.net.}}, inter-connecting an
  arbitrary number of Chord networks.

\end{itemize}
%
The final goal is to grasp the complete potential that co-located
nodes offer and to deepen the study of overlay networks'
inter-connection using these types of nodes.

\subsection{Outline}
%% Outline
The remainder of the paper is organized as follows: In
Section~\ref{sec:relatedwork}, we summarize the mechanisms proposed in
the literature related to the overlay networks' cooperation. In
Section~\ref{sec:protocol}, we introduce our Synapse protocol, by
means of example and pseudocode. In Section~\ref{sec:simulations}, we
present the results of our simulations of the Synapse protocol to
capture the behavior of key metrics traditionally used to measure the
efficiency of information retrieval. In Section~\ref{sec:deployement},
we describe a deployment of a client prototype\footnote{Freely
  available
  at~{\sf http://www-sop.inria.fr/teams/lognet/synapse}.} over the
Grid'5000 platform. Conclusions and further work conclude.



