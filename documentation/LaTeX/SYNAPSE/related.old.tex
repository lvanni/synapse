% related.tex

%% CT: ne dis pas ce que tu fais, pas la place, juste fais le :-)
%% CT: peut �tre plus appuyer aussi qu'une fa�on de collaborer, c de faire
%% CT: du merging (ce qui est couteux, citer EPFL, KTH) du hi�rarchique,
%% CT: ce qui suppose aussi de merger ou d'autres m�canismes de maintient de la topologie

Pointing out the limits of a unique global structured overlay
(rigidity, maintenance cost, security, $\dots$), several proposition
has been made over the years to build alternate topologies based on
the co-existance of smaller local overlay networks. A first approach
has been based on hierarchical systems~\cite{Biersack,XuMH03}, leading
to the requirement of costly merging mechanisms to ensure a high level
of exhaustiveness. In a more general view, merging several co-existing
structured overlay networks has been shown to be a very costly
operation~\cite{Datta,Haridi}, leading to inefficient overlay
networks~\cite{cheng2006tdh}.

We organize the related work into two parts: related mechanisms that
will enable/ease overlay
inter-connection %on top of the current Internet architecture
and we will briefly present some clean-slate routing architectures, as
discussed in \cite{feldmann_internet_2007}, built with networks
inter-connection %this
feature in mind from the ground up. Both parts share the same final
goal, that is, providing easier ways to inter-connect networks. We
could also cite some works that have been studying hierarchical DHT
systems~\cite{Biersack} which also consider multiple spaces and some
elected super peers promoted to a top-level overlay network. But the
main issue is that they introduce a multi-level addressing and lookup
space whereas we, in this work, try to avoid it in order to be more
generic. Hence we can say that our work subsumes hierarchical
DHTs.\newline

\textbf{Related mechanisms:} We can identify two main mechanisms for
enabling overlay networks inter-communication on top of the current
Internet architecture: \textit{co-located nodes} and
\textit{gateways}. The main difference between the two is that
co-located nodes fully participate in the routing process of the
various overlays they are registered to, whereas gateways are nodes
have only a specific pointer to another node in another overlay
networks and do not actively participate in the routing process of the
different overlays they are registered to. Co-located nodes have thus
higher state overhead than gateways, since they have to maintain more
pointers and process more messages due to their active participation.
Yet, nowadays it is more common to use multiple P2P applications in
the same time, and overlooking the possibility to exploit this would
be limiting.

%%
%%
% talk about each of them (in some details), underline their
% strong points and their drawbacks...
% Question : what do we bring that they don't have? answer => don't know...
%% Important : underline the difference between introducing novel
%% architectures (clean slate) and mechanisms that could work no
%% matter what the underlying architecture is. (ours is more a
%% mechanism)
%%
%%

Recently, authors in~\cite{cheng_bridging_2007}, stating that complete
merging is inefficient, propose a novel search protocol, based on
gateways called ``\textit{DHT-gatewaying}'', which is scalable and
efficient across homogeneous\footnote{Homogeneous DHTs:~same
  implementation and same keysize (ex. Two 160-bit Chord DHTs)},
heterogeneous\footnote{Heterogeneous DHTs:~same implementation and
  different keysize (ex. One 160-bit Chord and one 256-bit Chord
  DHTs)} and assorted\footnote{Assorted DHTs:~different implementation
  and/or different keysize (ex. One 160-bit Chord and one 256-CAN
  DHTs)}~co-existing DHTs. Their argument is that there isn't a
preferred DHT implementation, and that peers are members of
co-existing DHTs. Their assumptions are (\textit{i}) only some peers
support the implementations of different DHTs and (\textit{ii}) some
peers are directly connected to peers that are members of other DHTs,
and are called \textit{Virtual Gateways (VG))}. Their gatewaying
protocol works in the following way: when a request is sent in one
DHT, and no result was found, the requester can decide to widen his
search by forwarding its original search request to nodes which belong
to other DHTs (cross-DHT search). Those nodes will ``map'' the search
to the format which is supported by their relative DHTs. Once the
mapping is done, the search is carried out in each DHTs, and if a
result is found, it will be returned to the original requester. Note
that a Time-To-Live (TTL) value is added to the original search, in
order to avoid cycles; this value is decremented each time a request
crosses a new DHT domain. Because VGs can be overloaded, authors
devised a mechanism in order to distribute the mapping by electing
more VGs (according to a specific VG determination scheme), and they
also introduced self-organizing ``gateways pointers'' whose roles are
to keep track of VGs where-abouts. Conceptually, this work seems the
closest to our proposition. Our purpose is to give study more
accurately ...


%% what can i say about the lack of algorithms? since they provide a
%% detailed description of their protocol... + yes in the title it is
%% said 'Wireless'...but their protocol
% is applicable in wired networks too

%% \textbf{talk about : }\cite{furtado_multiple_2007}\\
Author in \cite{furtado_multiple_2007} presents mechanisms for
managing the multiple identifier spaces as well as inter-space linking
and routing alternatives. They consider multiple spaces with some
degree of intersection between spaces,~\ie~with co-located nodes. They
compared various inter-space routing policies by analyzing which
trade-offs, in terms of state overhead, would give the best results in
terms of the number of messages generated and routed, the number of
hops it takes to find a result and the state overhead (\ie~the number
of fingers a node has to keep). They do not present any algorithms but
they do provide an comparative analytical study of the different
policies. They showed that with some dynamic finger caching and with
multiple gateways (in order to avoid bottlenecks and single points of
failures) which are tactfully laid out, they attain pretty good
performances.


% \textbf{talk about : }\cite{cheng2006tdh}\\
In~\cite{cheng2006tdh} authors presented two models for two overlays
to be (de)composed, known as \textit{absorption}~(equivalent to
merging) and \textit{gatewaying}. Their protocol enables a CAN-DHT to
be completely absorbed into another one (in the case of the
absorption), and also provide a mechanism to create bridges between
DHTs (in the case of the gatewaying). They do not specifically take
advantage of a simple assumption that nodes can be part of multiple
overlays in the same time thus playing the role of natural bridges.
They did not evaluate their protocol and do not provide any algorithms
of their protocol.


%\textbf{talk about : }\cite{junjiro_design_2006}\\

Authors in \cite{junjiro_design_2006} present a model which considers
the symbiosis between different overlays networks with a specific goal
in mind: file sharing. They propose a mechanism for hybrid P2P
networks cooperation and investigates the influence of system
conditions such as the numbers of peers and the number of
meta-information a peer has to keep. Their work is bit more generic in
the sense that they do not focus on structured overlay networks as we
do, but still, they provide interesting observations on: \textit{(i)}~
joining a candidate network (\ie~considering to enhance one's QoS by
joining another network), \textit{(ii)}~selecting cooperative peers
(that is which peer(s) among this newly joined network will cooperate
with me), \textit{(iii)}~finding other P2P networks, \textit{(iv)}~the
very decision of starting cooperation, by taking into account the size
of the network (for instance a very large network will not really
benefit from a cooperation with a small network),
\textit{(v)}~relaying messages and files, \textit{(vi)}~caching
mechanisms in cooperative peers
 and finally \textit{(vii)}~ when it is appropriate to end a cooperation.
Their simulations showed the effect the popularity of a cooperative
peer on the search latency evaluation, that is the more a node has neighbors, the better,
as well as the effect of their caching mechanism which reduces (when appropriately
adjusted) the load on nodes (but interestingly does not contribute to faster search).

Authors in~\cite{kwon_synergy:overlay_2005} presents Synergy, an
overlay inter-networking architecture which improves the routing
performance in terms of delay, throughput and loss packets by
providing cooperative forwarding of flows. Authors acknowledge that
co-located nodes can serve as good candidates for enabling
inter-overlay routing and that they reduce traffic.

In this work, and in our previous preliminary work~\cite{LTB09}, we
also argue that co-located nodes are also good candidates for widening
the search capability. However in this paper we focus on the
co-located nodes heuristic in more details than the aforementioned
works by providing not only a simple algorithm which enables
inter-overlay routing but also more intensive simulations to show the
behaviours of such networks as well as a real implementation and
experiments. We first want to grasp the complete potential that
co-located nodes offer and we want to deepen the study
of overlay networks with these types of nodes.~\\

%%%%%%%%%%%%%%%%%%%%%%%
%%%%%%%%%%%%%%%%%%%%%%%
%%%%%%%%%%%%%%%%%%%%%%%

%% EXPLAIN HOW WE DO THINGS !!! AND HOW DO WE COMPARE TO THE OTHERS

%% second part : talk about other type of approaches ( (clean-slates)
%% internet routing architectures ) //Detour and NIRA

%% Partie ``clean-slate'' encore bcp trop longue. comme tu le dis
%% c pas directement reli� au notre. Je dirais de faire un paragraphe
%% que tu d�places avant la partie qui nous int�resse vraiment.

%% CT: cette phrase est bien, sers-t-en pour appuyer le fait que
%% CT: le futur internet passera par de l'interconnection
%% CT: (mais au d�but du related work)

\textbf{Clean-slate routing architectures:}~The following works,
although not directly related to ours, propose alternatives to the
current Internet architecture and also present interesting methods for
inter-connecting domains.

%%ROFL~\cite{caesar_rofl:_2006}:
%% briefly explain what is their contribution

Authors in \cite{caesar_rofl:_2006} propose and analyze a routing
scheme based on flat names. They want to get rid of location
information that we can find the network layer and route directly on
the identities themselves. Although they propose a compact routing
scheme, some questions arise regarding the scalability of their
solution.

%% CT: oui oui, les overlays c pas bien pour les perfs, je suis
%% CT: d'accord, nous ce qu'on fait, c'est de la recherche d'information
%% CT: et entre nous, on est pas les seuls � faire des overlays pour �a :-)

%% briefly explain what is their contribution
In~\cite{yang_nira:new_2003} authors present the design of a new
Internet routing architecture (NIRA) that aims at providing end users
the ability to choose the sequence of Internet service providers a
packet traverses at the domain level, ~\ie~they will be able to choose
how inter-domain routing is done.
%% authors VS overlays
Authors argue that overlay networks are not ubiquitous, that only
nodes on the overlay network can control the packet's paths by
tunnelling traffic through other nodes on the overlay. They also
present scepticisms regarding the scalability of the overlay,
stipulating that they are unlikely to scale up so to include every
user on the Internet, and that an overlay path may traverse duplicate
physical links.
% Feedback-Based Routing~\cite{zhu_feedback_2003}:
%% briefly explain what is their contribution

%% CT: voil� pareil, bien s�r qu'Overnet ne marche pas si le cable est pas branch�
%% CT: ou si y'a une coupure de courant.
%% CT: je n'ai pas lu les papiers, mais j'ai l'impression que les auteurs disent
%% CT: "pour faire de l'interconnexion s�re et efficace de composants r�seaux,
%% CT: il ne faut pas se placer au niveau overlay, mais en dessous" et je suis d'accord.
%% CT: mais nous on met en place un algorithme g�n�rique pour la recherche d'information
%% CT: dans des DHT qui sont interconnect�s... pas pareil.

In~\cite{zhu_feedback_2003} authors propose a routing scheme which
separates structural information and dynamic information. They provide
a system in which only structural information is disseminated, and
dynamic information is discovered by routers based on feedbacks and
probes, which apparently helps improving the routing decisions.
%% authors VS overlays
Authors believe that overlay network is not the final solution for
reliable packet forwarding. Their reasoning is based on the fact that
overlay network only increase the probability that the communication
does not fail when there are only isolated routing failures in the
network. No overlay network is going to function when the underlying
routing infrastructure completely fails.

%%DETOUR~\cite{savage_detour:case_1999}:
%% briefly explain what is their contribution

%% Explain their common viewpoint that the BGP protocol is not enough and much
%% flexibility has to be given to the user concerning the route of packets
%% when traversing domains

%\textbf{Common ascertainment:}

%% CT: Pareil, BGP c'est niveau 3, nous on s'occupe de r�seaux logiques, niveau applicatif,
%% CT: une seule de nos DHT peut �tre distribu� sur plein d'AS, qui vont utiliser
%% CT: RIP, BGP ou OSPF pour router entre eux, c m�me pas notre probl�me.
%% CT: Nous, on cherche pas � router entre des AS, on veut
%% CT: faire un lookup sur plusieurs DHTs.

Regarding the clean-slate redesigns of the Internet, most of the cited
authors (\cite{yang_nira:new_2003},\cite{zhu_feedback_2003}) seem to
agree that the BGP routing protocol~\cite{rfc1771}, the main protocol
for inter-domain routing, does not provide enough information
regarding the packet routes and does not give the possibility to the
users to be able to choose their own domain-level routes. BGP does not
scale particularly well, converges rather slowly (and sometimes with
certain routing policy combinations it diverges
\cite{labovitz_delayed_2000}). They attempt, and so do we, to
circumvent the current Internet limitations by proposing an
alternative method for interconnecting networks.

Although their insights and proposals are more than relevant, we do
believe they are far from being applicable in practise. The obvious
reason is that the current established Internet cannot be changed in
such radical ways and their solutions cannot be easily deployed. In
this sense, overlay networks are a more flexible solution than
complete re-designs, plus they can also serve as a framework for
clean-slate re-designs to accelerate prototyping their new approaches.

In this work we focus our attention on inter-connecting overlay
networks, because we believe that since their introduction they have
matured and they can answer most of todays Internet's challenges. We
provide what we consider as a simple and natural solution for bridging
overlay networks together.

\begin{comment}

  In this sense, and in response to overlay detractors, we argue
  %% list some coherent responses for their comments that ON are not
  %% suited for interconnecting domains
  that works like \cite{XuMK03}, \cite{EURECOM+1205} and \cite{zhou_balancing_2003}
  show  efficient method for constructing an overlay network while taking into
  account the underlying topology. Therefore we can say with confidence that we do
  have mechanisms in order to ensure that the paths the packets traverse
  are not using duplicate physical links.

\end{comment}

%% SPECIAL MENTION OF THE P2PSIP effort going on at the IETF



